\documentclass[UTF8]{ctexart}

\usepackage{amsmath}
\usepackage{amssymb}
\usepackage{amsthm}
\usepackage{graphicx}
\usepackage{CJK}
\usepackage{float}
\usepackage{mdframed}
\providecommand{\abs}[1]{\lvert#1\rvert}
\providecommand{\norm}[1]{\lVert#1\rVert}
\providecommand{\ud}[1]{\underline{#1}}

\newmdtheoremenv{thm}{Theorem}
\newmdtheoremenv{lemma}[thm]{Lemma}
\newmdtheoremenv{fact}[thm]{Fact}
\newmdtheoremenv{cor}[thm]{Corollary}
\newtheorem{eg}{Example}
\newtheorem{ex}{Exercise}
\newmdtheoremenv{defi}{Definition}
\newenvironment{sol}
  {\par\vspace{3mm}\noindent{\it Solution}.}
  {\qed \\ \medskip}

\newcommand{\ov}{\overline}
\newcommand{\ca}{{\cal A}}
\newcommand{\cb}{{\cal B}}
\newcommand{\cc}{{\cal C}}
\newcommand{\cd}{{\cal D}}
\newcommand{\ce}{{\cal E}}
\newcommand{\cf}{{\cal F}}
\newcommand{\ch}{{\cal H}}
\newcommand{\cl}{{\cal L}}
\newcommand{\cm}{{\cal M}}
\newcommand{\cp}{{\cal P}}
\newcommand{\cs}{{\cal S}}
\newcommand{\cz}{{\cal Z}}
\newcommand{\eps}{\varepsilon}
\newcommand{\ra}{\rightarrow}
\newcommand{\la}{\leftarrow}
\newcommand{\Ra}{\Rightarrow}
\newcommand{\dist}{\mbox{\rm dist}}
\newcommand{\bn}{{\mathbb N}}
\newcommand{\bz}{{\mathbb Z}}

\newcommand{\expe}{{\mathsf E}}
\newcommand{\pr}{{\mathsf{Pr}}}


\setlength{\parindent}{0pt}
%\setlength{\parskip}{2ex}
\newenvironment{proofof}[1]{\bigskip\noindent{\itshape #1. }}{\hfill$\Box$\medskip}

\theoremstyle{definition}
\newtheorem{problem}{Problem}
\newtheorem*{problem*}{Problem}

\pagenumbering{gobble}

\begin{document}

\title{CS477 Combinatorics: Homework 1}
\author{于峥 518030910437}
\date{Mar. 4, 2020}

\maketitle

\begin{problem}
令$A = [5]$, $B = [3]$, $C = [3]$。

(a) 列举一个$(A^B)^C$的元素。

(b) 列举一个$A^{B\times C}$的元素。
\begin{sol}

  (a) $f: C \rightarrow A^B$, $\forall x\in C, f(x) = g$, 
  其中$g:B\rightarrow A$, $\forall x\in B, g(x)=1$。

  写成有序组的形式 $((1,1,1),(1,1,1),(1,1,1))$。

  (b) $f: B \times C \rightarrow A$, $\forall x \in B \times C$, $f(x)=1$,

  写成矩阵的形式
  $$
  \begin{bmatrix}
    1 & 1 & 1 \\
    1 & 1 & 1 \\
    1 & 1 & 1 
  \end{bmatrix}
  $$

\end{sol}

\end{problem}

\begin{problem}
在空间中有$n$个不同的蓝点和$m$个不同的红点,在每个蓝点和每个红点的中点画上一个紫色的点。(紫色的点有可能和某个红点或者蓝点重合。)

(a) 求紫色点的数量的最小值。

(b) 给出并证明上述取到最小值时的蓝点和红点分布的充要条件。
\begin{sol}
  (a) 蓝色点集合为$A$, 红色点集合为$B$, 由于中点公式容易发现
  紫色点数量$=|A+B|$。注意到如果把红点或蓝点整体平移,紫点的数量不会改变。所以我们通过平移
  让这些点两两不同。
  
  我们可以给过每个点做一条直线使得这些直线互相平行,由于点是有限的,
  能够找到某个斜率,让每条直线只经过一个点,我们作这些直线的垂线为x轴,随意选定方向和原点。
  我们把每个点映射到点所在的x坐标,那么所有点对应的x坐标互不相同。
  
  将红蓝点映射到$x$坐标形成的集合记为$A'$, $B'$, 下证 $|A'+B'|\geq m+n-1$。
  
  
  我们知道实数集合加法满足$|A+B|\geq|A'+B'|\geq m+n-1$,因此至少有$m+n-1$个紫点。
  $$
  \begin{aligned}
    A' = \{a_1, a_2, a_3, \dots, a_n\} \\
    B' = \{b_1, b_2, b_3, \dots, b_m\} \\
    a_1 < a_2 < \dots < a_n \\
    b_1 < b_2 < \dots < b_m
  \end{aligned}
  $$
  那么有$a_1 + b_1 < a_2 + b_1 < \dots < a_n + b_1 < a_n + b_2 < \dots < a_n + b_m$.
  这里已经有$n+m-1$个数, 所以集合的和大于$n+m-1$。

  (b) 红点和蓝点分别位于两条平行直线上,在直线上等距分布,并且相邻蓝点的距离与相邻红点相同。或者至少有一种颜色的点只有一个。
  
  \textbf{必要性} : 代入验证发现可取到最小值。

  \textbf{充分性} :
  只考虑$n>1,m>1$的情况,思考上一题证明中等号取到的条件, 对于$a_{n-1}+b_2$必与
  $a_1 + b_1 , a_2 + b_1 , \dots , a_n + b_1 , a_n + b_2 , \dots , a_n + b_m$
  中某个数相同,我们将这些数写成矩阵形式,
  $$
  \begin{bmatrix}
    a_1 + b_1 & a_2 + b_1 & \dots & a_n + b_1 \\
    a_1 + b_2 & a_2 + b_2 & \dots & a_n + b_2 \\
    \dots     &     \dots & \dots & \dots     \\
    a_1 + b_m & a_2 + b_m & \dots & a_n + b_m \\
  \end{bmatrix}
  $$
  显然$a_{n-1}+b_2$必须要与所在位置右上的数相同,因为这个矩阵行列都是单调递增的。进而一次考虑
  其他的数可以发现每个反斜线上的数都是相等的。进而可以得出这样的结论,
  若将$a_i$,$b_i$看成数列,那么这两个数列都是等差数列,且差相同。注意到$a_i$,$b_i$分别为红蓝点
  的横坐标,对于横坐标加和相同的红蓝点对,纵坐标加和也必须满足,否则点数必将大于$m+n-1$。因此
  我们可以得到点的分布。

  
\end{sol}
\end{problem}

\begin{problem}
证明:对任意的自然数$n$,
\[
\sum_{r=0}^n r \binom{n}{r} = n 2^{n-1}.
\]
\begin{sol}
这个公式右边的组合意义为从$n$个人中选一个队长, 然后从剩下的$n-1$个人中选择队员。
而左边的式子则是先枚举队伍中有几个人,选择了队伍中的人后再选择队长。

归纳法:$n=0,1$时容易验证等式成立,$n>1$时
$$
\begin{aligned}
  \sum_{r=0}^n r \binom{n}{r} 
  &= n + \sum_{r=1}^{n-1}r\binom{n-1}{r}+r\binom{n-1}{r-1}\\
  &= n + (n-1)2^{n-2}+\sum_{r=0}^{n-2}(r+1)\binom{n-1}{r}\\
  &= (n-1)2^{n-2}+\sum_{r=0}^{n-1}(r+1)\binom{n-1}{r}\\
  &= (n-1)2^{n-2}+(n-1)2^{n-2}+2^{n-1}\\
  &= n2^{n-1}
\end{aligned}
$$

\end{sol}
\end{problem}

\begin{problem}
考虑所有的有序组$\alpha = (A_1, A_2, \dots, A_k)$,其中$A_i \subseteq [n]$,
定义$S(\alpha) = |A_1 \cup A_2 \dots \cup A_k|$。所有这样的有序组构成的集合为$\mathcal{U}$。

(a) 计算$|\mathcal{U}|$。

(b) 计算
\[
\sum_{\alpha \in \mathcal{U}} S(\alpha).
\]

\begin{sol}
  (a) $A_i$有$2^n$种选择, 那么$\alpha$有$2^{nk}$种选择。$|\mathcal{U}|=2^{nk}$。

  (b) 考虑有多少个$\alpha$满足$S(\alpha)=i$, 首先$A_1 \cup A_2 \dots \cup A_k$有
  $\binom{n}{i}$种可能,对于其中的每一个元素,在$A_1,A_2,\cdots,A_k$中至少出现一次,所以有
  $2^k-1$中可能,则存在$(2^k-1)^i\binom{n}{i}$个$\alpha$满足$S(\alpha)=i$。

  我们还可以从另一种角度来思考这个问题,我们考虑$[n]$中的每个元素被计入了答案几次,首先指定
  一个元素,只要它在$A_1,A_2,\cdots,A_k$中至少出现一次, 那么它就有1的贡献,而其他元素的出现与否不影响
  它对答案的贡献,所以每个$[n]$中的元素都有$(2^k-1)2^{k(n-1)}$的贡献, 则得出
  $$
  \sum_{\alpha \in \mathcal{U}}S(\alpha)=\sum_{i=0}^n i(2^k-1)^i\binom{n}{i}=(2^k-1)n2^{k(n-1)}
  $$
\end{sol}
\end{problem}

\begin{problem}
在
\[
\binom{2020}{0}, \binom{2020}{1}, \dots, \binom{2020}{2020}
\]
中有多少个奇数?
\begin{sol}
  $\binom{n}{0},\binom{n}{1},\cdots,\binom{n}{n}$中奇数的数量取决于$n$的二进制位上$1$的个数,
  证明这一点可以使用Lucas定理,$p$为素数,则有
  $$
  \begin{aligned}
    \binom{n}{m}=&\prod_{i=0}^k\binom{m_i}{n_i}\pmod p\\
    m =& m_kp^k+m_{k-1}p^{k-1}+\cdots+m_1p+m_0\\
    n =& n_kp^k+n_{k-1}p^{k-1}+\cdots+n_1p+n_0\\
  \end{aligned}  
  $$

  可以取$p$为$2$,则$m_i$为0时,$n_i$必须为0才能为奇数,那么只有$n$ \& $m = m$时$\binom{n}{m}$为奇数。

  $2020=11111100100_2$, 有$2^7$个奇数。
\end{sol}
\end{problem}

\begin{problem}(*)
接课上的定义

(a) 证明平面上至多可以画可数个8。

(b) 证明平面上至多可以画可数个Y。
\begin{sol}

  (a) 对于平面上的每个8,我们可以从8的两个o中各任取一个有理点$A,B$,组成集合(无序点对)$\{A,B\}$, 来表示这个8,
  由于8不能相交,所以不可能有点对能同时表示两个8。由于有理点的可数性,这样的无序点对也可数,进而8也可数。
  
  (b) 值得注意的是,曲线上不一定存在有理点,所以无法用上题的方法。

  我们把Y的中心点记为O, 三条线段的终点为$a,b,c$, 三点绕$O$点方向为逆时针。
  我们找到三个不相交有理圆(圆心为有理点,半径为有理数),圆心分别为$A, B, C$。并且这三个圆
  分别不与Y的另外两条线相交。

  假设有两个Y($Y_1, Y_2$)共用三个相同的有理圆,我们修改两个Y伸出的三条线,
  这三条线碰到圆之后直接连向圆心,那么$AY_1BY_2$构成闭环,由于$a_1, b_1, c_1$,
  $a_2, b_2, c_2$都必须是逆时针,可以观察到$c_1$. $c_2$不可能都在
  $AY_1BY_2$这个闭环内,这意味这圆C必然与这个闭环相交,则产生矛盾。

  所以不会有两个Y共用三个相同的有理圆。而有理圆是可数的,所以Y是可数的。

  我们可以容易的用第二问的结论证明第一问, 首先容易用相同方法证明平面上只能放
  可数个+,可以把8的两个圈各剪开一个缝,那么8的放置就是+的放置的特殊情况,因此
  8是可数的。

\end{sol}
\end{problem}

\end{document}

