\documentclass[UTF8]{ctexart}
\usepackage{amsmath}
\usepackage{amssymb}
\usepackage{amsthm}
\usepackage{graphicx}
\usepackage{CJK}
\usepackage{float}
\usepackage{mdframed}
\providecommand{\abs}[1]{\lvert#1\rvert}
\providecommand{\norm}[1]{\lVert#1\rVert}
\providecommand{\ud}[1]{\underline{#1}}

\newmdtheoremenv{thm}{Theorem}
\newmdtheoremenv{lemma}[thm]{Lemma}
\newmdtheoremenv{fact}[thm]{Fact}
\newmdtheoremenv{cor}[thm]{Corollary}
\newtheorem{eg}{Example}
\newtheorem{ex}{Exercise}
\newmdtheoremenv{defi}{Definition}
\newenvironment{sol}
  {\par\vspace{3mm}\noindent{\it Solution}.}
  {\qed \\ \medskip}

\newcommand{\ov}{\overline}
\newcommand{\ca}{{\cal A}}
\newcommand{\cb}{{\cal B}}
\newcommand{\cc}{{\cal C}}
\newcommand{\cd}{{\cal D}}
\newcommand{\ce}{{\cal E}}
\newcommand{\cf}{{\cal F}}
\newcommand{\ch}{{\cal H}}
\newcommand{\cl}{{\cal L}}
\newcommand{\cm}{{\cal M}}
\newcommand{\cp}{{\cal P}}
\newcommand{\cs}{{\cal S}}
\newcommand{\cz}{{\cal Z}}
\newcommand{\eps}{\varepsilon}
\newcommand{\ra}{\rightarrow}
\newcommand{\la}{\leftarrow}
\newcommand{\Ra}{\Rightarrow}
\newcommand{\dist}{\mbox{\rm dist}}
\newcommand{\bn}{{\mathbb N}}
\newcommand{\bz}{{\mathbb Z}}

\newcommand{\expe}{{\mathsf E}}
\newcommand{\pr}{{\mathsf{Pr}}}


\setlength{\parindent}{0pt}
%\setlength{\parskip}{2ex}
\newenvironment{proofof}[1]{\bigskip\noindent{\itshape #1. }}{\hfill$\Box$\medskip}

\theoremstyle{definition}
\newtheorem{problem}{Problem}
\newtheorem*{problem*}{Problem}

\pagenumbering{gobble}

\begin{document}

\title{CS477 Combinatorics: Homework 10}
\author{于峥 518030910437}
\date{\today}

\maketitle

\begin{problem}
证明:任何一个元素为非负实数、每行每列之和为1的$n \times n$矩阵可以表示成置换矩阵的凸组合。
\begin{sol}
  首先我们不难把矩阵转化为二分图,如果$M_{ij}\not=0$, 那么二分图$i,j$对应的点之间有边相连。
  我们首先证明这个二分图$G=(L\cup R, E)$是有完美匹配的,即该二分图满足Hall条件。

  如果存在子集$S\subset L$, 满足$|\mathcal{N}(S)|<|S|$, 注意到$|S|$所对应
的行上的数字和为$|S|$。这是$|\mathcal{N}(S)|$对应列上的数字的部分和,而$|\mathcal{N}(S)|$对应列
上的数字和为$|\mathcal{N}(S)| < |S|$, 产生矛盾,因此必定所有集合都满足Hall条件,即二分图
有完美匹配。
因此矩阵上至少有$n$个非零的$M_{ij}$,我们可以对矩阵中不为0的$M_{ij}$个数进行归纳。

\paragraph{Induction Base} 如果有$n$个位置非零,这$n$个位置必定都为1,不难发现
$M$本身就是个置换矩阵。

\paragraph{Induction Step} 对于非零位置数大于$n$的矩阵,我们找到对应二分图上的一个完美匹配,
选择将匹配边对应的矩阵中的数中最小的一个,设为$\lambda_0$, 匹配对应的矩阵为$P_0$, 
那么我们将矩阵减去$\lambda_0P_0$非零位置位置个数必定至少减少$1$,因此我们将剩下的矩阵
可以看作$(1-\lambda_0)M'$, $M'$依然是每行每列都为1的矩阵,根据归纳$M'$可以表示成
置换矩阵的凸组合。

那么考虑此时矩阵$M=\lambda_0P_0+(1-\lambda_0)M'=\lambda_0P_0+(1-\lambda_0)\sum_{i=1}^k\lambda_kP_k$。
并且$\lambda_0+(1-\lambda_0)\sum_{i=1}^k\lambda_k=1$,因此$M$可以表示成置换矩阵的凸组合。


\end{sol}
\end{problem}

\begin{problem}
$G$是一个二分图,最大度数为$\Delta$。

证明:可以对$G$的边用$\Delta$种颜色染色,使得任何两条有公共点的边不同色。

\begin{sol}
  假设我们至少需要$C$种颜色才能使得任何两条有公共点的边不同色。那么我们显然
  有$C \geq \Delta$。

  考虑二分图左侧$L$所有度数为$\Delta$的点的集合$S$, 那么$S$和$\mathcal{N}(S)$必然存在
  大小为$|S|$的匹配,因为考虑任意子集$T \subset S$, $T$和$\mathcal{N}(T)$之间的边数
  为$|T|\Delta = |\mathcal{N}(T)|\overline{d}$. $\overline{d}$ 是$\mathcal{N}(T)$
  的平均度数显然小于等于$\Delta$, 所以$|T| \leq |\mathcal{N}(T)|$,这意味着这个子图满足Hall
  条件。我们记这个匹配为$M_L$. 同理对右侧拥有度数$\Delta$的点的集合找到完备匹配$M_R$.

  那么 $M_L \cup M_R$形成一个子图,首先我们有这个子图中的点度数都小于等于2。所以这个子图的每个连通块
  只能是链或者环。  

  i) 如果是环,我们可以做到选泽环上一半的边, 这些边之间没有公共点,我们把它们从图上删去。
  此时这个连通块中的每个点度数都减少了$1$。

  ii) 如果是链,若这条链的长度为奇数$2k+1$,那么我们可以做到删去$k+1$条互相没有公共点的边使得每个点度数都减少$1$。
  
  若是偶数,不难发现不可能链条两端的点度数在原图都是当前图的最大度数$\Delta$, 否则会出现两个度数为$\Delta$的点匹配
  到了同一个点,根据构造这是不可能的。所以如果我们发现链的一端的度数小于$\Delta$, 那我们就从另一端
  开始,每隔一条边就删去一条边。我们可以做每个度数为$\Delta$的点度数都减少$1$。

  每次操作可以让最大度数减少1,因此$\Delta$次后图将没有边。我们让一次删去的边染上同一种
  颜色,我们就得到了一个大小为$\Delta$ 的染色。
\end{sol}
\end{problem}

\begin{problem}
证明:对任意整数$k \geq 2$,
\[
r(3, k) \leq \frac{k^2+3}{2}.
\]
\begin{sol}
  \paragraph{Induction Base} 由于$r(3,2)=3,r(3,3)=6$可以验证是满足要求的。
  \paragraph{Induction Step} 假设对于小于$k$时都成立,下面证明对于$k$也成立。

  我们可以证明,$r(3,k)\leq r(3,k-1)+k$, 因为我们可以选择一个点,对于它连出去的
  $r(3,k-1)+k-1$条边,如果黄色边数量大于等于$k$, 那么对于黄色边对应的点,如果不全是
  蓝色(存在蓝色$K_k$)那么必有黄色三角形。如果黄色边数量小于$k$,那么就至少存在$r(3,k-1)$
  条蓝色边,我们知道这些边对应的点如果存在蓝色$K_{k-1}$, 那么加上选择的点构成蓝色$K_k$。

  当$k$是奇数时
  $$
  r(3,k)\leq k + \lfloor\frac {(k-1)^2+3}2\rfloor =
  k+\frac{(k-1)^2+2}2=\frac{k^2+3}{2}
  $$

  当$k$是偶数时,我们可以证明$\frac {k^2+2} {2}$个点将足够。因为如果每个点连出去
  的蓝边数量都小于$\frac {(k-1)^2+3} 2$,并且黄边数量小于$k$。

  那么每个点连出去的黄边数量必须为$k-1$,而$k-1$是奇数,而$\frac {k^2+2}2$也是奇数,
  因此这个黄色子图不满足握手定律。因此必然存在一个点满足蓝边大于等于$r(3,k-1)$或黄边
  大于等于$k$。因此$k$为偶数时也满足。
  
\end{sol}
\end{problem}

\begin{problem}
定义$r_k(3)$为最小的$N$使得对$K_N$的边任意$k$染色,总有一个同色三角形。证明:
\[
2^k < r_k(3) \leq \lfloor e k! \rfloor + 1.
\]

\begin{sol}
  a) $2^k < r_k(3)$

  \texttt{Induction Base} 当$k=1$时,显然2个点不可能有同色三角形。

  \texttt{Induction Step} 当小于$k$都成立时, 我们取两个大小为$2^{k-1}$的图且
  两个图没有同色三角形,根据归纳这能做到。我们用新的第$k$种颜色把两个图之间的边连起来,
  就得到了大小为$2^k$且没有同色三角形的图。

  b) $r_k(3) \leq \lfloor e k! \rfloor + 1$

  用$k$种颜色对$K_n$染色,考虑$K_n$中某个点,它的邻边中必然有某种颜色出现了$\lceil \frac {n-1} k \rceil$次,
  不妨设为黄色,如果这对应的$\lceil \frac {n-1} k \rceil$个点的子图中出现了黄色,那么意味着出现了
  一个黄色三角形。否则如果$\lceil \frac {n-1} k \rceil \geq r_{k-1}(3)$,那么也有同色三角形。

  因此我们得到
  $$
  r_k(3) \leq k(r_{k-1}(3)-1) + 2
  $$
  又$r_2(3)=3$, 我们让$f_1=3, f_k=kf_{k-1}-k+2$. 那么我们可归纳证明
  $$
  f_k - 1 = \sum_{i=0}^k\frac{k!}{i!}
  $$
  当$k=1$时验证成立。若当小于$k$时成立,那么我们有$k$时也成立
  $$
  f_k-1=k(f_{k-1} - 1) + 1 = k \sum_{i=0}^{k-1}\frac {(k-1)!} {i!} + \frac {k!}{k!}=\sum_{i=0}^k\frac{k!}{i!}
  $$
  因此$r_k(3) \leq f_k < ek! + 1$, 考虑$r_k(3)$为整数所以 $r_k(3) \leq \lfloor e k! \rfloor + 1$.


\end{sol}
\end{problem}

\begin{problem}
称一个关于$x_1, x_2, \dots, x_n$的方程$E$是Ramsey的,如果:对任意的$k$,存在一个$N$,使得
对任意的染色$f: [N] \to [k]$,总有$E$的同色解,即$E$的一组解$(a_1, a_2, \dots, a_n)$使得
$f(a_1) = f(a_2) = \dots = f(a_n)$。注意其中$a_i$不必两两不同。
对下面的方程的每一个,证明或否定它是Ramsey的。

(a) $E_1 : x = 2y$;

(b) $E_2 : x + y = z$;

(c) $E_3 : 5x + 3y = 7z + 12w$;

(d) $E_4 : x + 2020y = z$。

\begin{sol}
  (a) $E_1$ 不是Ramsey的,因为图$G=([N], \{\{x,y\}:x=2y\})$由若干条链组成,可以二染色,
  所以对于任意的$N$都能做到不出现相邻同色节点,因此没有同色解。

  (b) $E_2$是Ramsey的,$\forall k$, 考虑
  $$
  N = r_k(3) = \underbrace{r(3,3,\cdots,3)}_{\texttt{k个3}}
  $$
  对于染色$f:[N]\rightarrow[k]$, 我们考虑$K_N$, 对边进行$k$染色,染色方式是对于
  边$\{u,v\}$, 颜色为$f(|u-v|)$。那么由于其中必定存在同色三角形。设三角形的三个
  点为$a,b,c(a>b>c)$, 那么边所对应的数为$a-b,b-c,a-c$。由构造知道这三个数是同色的。
  又$(a-b)+(b-c)=a-c$,所以满足$E_2$。

  (c) 否定,对于$k=22$, 我们定义染色$f : \mathbb{Z}_+ \rightarrow [22]$, 
  $f(23^k\cdot j)=j \mod 23$,其中$23 \not| j$。
  假设存在一组同色解$f(x)=f(y)=f(z)=f(w)=j$.

  所以我们有
  $$
  5 \cdot 23^{k_1}(23x_1 + j) + 3 \cdot 23^{k_2}(23y_1 + j) = 
  7 \cdot 23^{k_3}(23z_1 + j) + 12 \cdot 23^{k_4}(23w_1 + j)
  $$

  我们将等式两边除以$23^{\max\{k_1,k_2,k_3,k_4\}}$

  $$
  5 \cdot 23^{m_1}(23x_1 + j) + 3 \cdot 23^{m_2}(23y_1 + j) = 
  7 \cdot 23^{m_3}(23z_1 + j) + 12 \cdot 23^{m_4}(23w_1 + j)
  $$

  然后对等式两边对$23$取模, 注意到根据定义,$j \not= 0$, 然后除以$j$
  $$
  5 \cdot 23^{m_1} + 3 \cdot 23^{m_2} = 
  7 \cdot 23^{m_3} + 12 \cdot 23^{m_4} \pmod {23}
  $$

  由于$m_1,m_2,m_3,m_4$中至少有一个是$0$, 而枚举所有情况检查后发现不存在合法的
  $m_1,m_2,m_3,m_4$, 所以不存在同色解。
  
  (d) $E_4$是Ramsey的,借助范德蒙数的存在性来证明,
  \begin{lemma}
    对任意的$k, l$, 存在$N$, 使得$[N]$被$k$染色后中一定存在同色的长度为$l$的等差数列。
    最小的$N$记为$W(k,l)$。
  \end{lemma}
  注意到对于$w(k,2021)$,如果我们能保证等差数列的公差也和等差数列同色。那么我们就找到了方程
  $x+2020y=z$的一组解。我们记满足这样要求的最小的$N$为$f(k)$。

  我们归纳证明$f(k)$是存在的,首先$f(1)=2021$。

  然后我们可以证明
  $$
  f(k) \leq W(k, 2020f(k-1) + 1)
  $$

  因为对$[W(k, 2020f(k-1) + 1)]$进行$k$染色后, 其中存在一个长度为$2020f(k-1) + 1$
  的同色等差数列$a,a+d, \cdots, a + 2020f(k-1)d$。不妨设颜色为黄色,那么我们再考虑
  $d, 2d, \dots, f(k-1)d$, 如果这些数中存在黄色,设为$jd$,那么
  $a, a+jd, \cdots, a + 2020jd$就是满足要求的。否则$d, 2d, \dots, f(k-1)d$
  是被$k-1$染色的,根据归纳同样满足要求。

  所以对于任意的$k$, 都存在满足条件的$N$。
\end{sol}
\end{problem}

\end{document}
