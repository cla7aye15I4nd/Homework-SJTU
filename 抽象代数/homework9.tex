\documentclass{ximera}
\usepackage[UTF8]{ctex}
\usepackage{graphicx}

\title{MA150 Algebra}
\author{于峥}

\begin{document}
\begin{abstract}
    homework 9
\end{abstract}

\maketitle

\begin{problem} Page 110-1
    \begin{solution}
        \textbf{必要性}:若Sylow p子群$P$是$G$的正规子群,即$\forall g \in G$,$gPg^{-1}=P$。假设存在另一个Sylow p 
        子群$P^{'} \not=P$。由于任意两个Sylow p子群互相共轭,所以$\exists g \in G, P^{'}=gPg^{-1}=P$,则$G$有唯一的Sylow p子群。

        \textbf{充分性}:若$G$有唯一的Sylow p子群, 则显然$\forall g \in G,gPg^{-1}$是Sylow p子群。
        因为首先$\forall a, b \in P, a\not=b,gag^{-1}gbg^{-1}=gabg^{-1}\in gPg^{-1}$,所以$gPg^{-1}$是$G$的子群。
        其次$gag^{-1} \not gbg^{-1}$, 所以$|gPg^{-1}|=|P|$。
        所以$gPg^{-1}=P$, 所以$P$是$G$的正规子群。
    \end{solution}
\end{problem}

\begin{problem} Page 110-2
    \begin{solution}
        $$
            N(P) = \{g \in G | gPg^{-1} = P \}
        $$
        根据习题2-2(9),可知$P$是$N(P)$的正规子群,$N(P)$是$G$的子群,设$P=p^{r}$。

        所以$N(P)=p^r\cdot m$,且$\gcd(m,p)=1$,则$P$是$N(P)$唯一的Sylow p子群。

        i) $\forall g \in N(P)$, $gN(P)g^{-1}=N(P)$,所以$N(P) \subseteq N(N(P))$。
        
        ii) $\forall g \in N(N(P))$, $gN(P)g^{-1}=N(P)$。因为$P\in N(P)$,
        则$gPg^{-1} \subseteq gN(P)g^{-1} = N(P)$,所以$gPg^{-1}$是N(P)的Sylow p子群。
        根据上一题的结论,$gPg^{-1}=P$。所以$g\in N(P)$,$N(N(P)) \subseteq N(P)$。

        综上,$N(P)=N(N(P))$。
    \end{solution}
\end{problem}

\begin{problem} Page 110-3
    \begin{solution}
        $|S_4|=4!=2^3\cdot 3$。$n_2 = 2t+1| 3$, 直接枚举所有子群发现有三个8阶子群。
        $$
        \begin{aligned}
            N_1&=\{(1),(1234),(13)(24),(1432),(13),(12)(34),(24),(14)(23) \} \\
            N_2&=\{(1),(1324),(12)(34),(1423),(12),(13)(24),(34),(14)(32) \} \\  
            N_3&=\{(1),(1243),(14)(23),(1342),(14),(12)(43),(23),(13)(24) \}
        \end{aligned}
        $$
    \end{solution}
\end{problem}

\begin{problem} Page 110-4
    \begin{solution}
        $|A_4|=|S_4|/2=12=2^2\cdot 3$。$n_2 = 2t+1 | 3$。枚举发现有1个
        $$
            N=\{(1),(12)(34),(13)(24),(14)(23)\}
        $$
    \end{solution}
\end{problem}

\begin{problem} Page 110-6
    \begin{solution}
        $|S_5|=5 \cdot 4!$。$n_5 = 5t+1 | 24 \rightarrow n_5 = 1$或$n_5=6$。

        不难想到大小为5的轮换所生成的群大小为$5$,这样的群有$\frac {4!} 4 = 6$个,所以
        $n_5 = 6$。

        例如
        $$
        \begin{aligned}
            N_1&=\{(1),(12345),(13524),(15432),(14253)\}\\
            N_2&=\{(1),(12354),(13425),(14532),(15243)\}
        \end{aligned}
        $$
    \end{solution}
\end{problem}

\begin{problem} Show that a group with order 145 is a cyclic group.
    \begin{solution}
        令群$|G|=145=5 \cdot 29$。则$G$有Sylow 5子群和Sylow 7子群。设为$H,K$。
        且$H,K$都是循环群,令$H=\langle a \rangle, K = \langle b \ rangle$。又
        $$
        \left \{
        \begin{aligned}
            n_5 = 5t + 1 | 29 \\
            n_{29} = 29s + 1 | 5
        \end{aligned}
        \right.
        \Longrightarrow
        \left \{
        \begin{aligned}
            n_5 = 1 \\
            n_{29} = 1
        \end{aligned}
        \right.
        $$
        所以$H,K$为正规子群,即$\forall h \in H, k \ni K$, $hk=kh$, 所以$ord(ab)=145$,$G=\langle ab \rangle$为循环群。
    \end{solution}
\end{problem}

\end{document}