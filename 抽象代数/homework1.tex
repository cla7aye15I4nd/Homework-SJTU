\documentclass{ximera}
\usepackage[UTF8]{ctex}

\title{MA150 Algebra }
\author{于峥}

\begin{document}
\begin{abstract}
    homework 1
\end{abstract}
\maketitle

\begin{problem} Page 5-4
    \begin{solution}
        (1) 反身性: $\phi(a) = \phi(a) \rightarrow a \sim a$。

        (2) 对称性: $a \sim b \rightarrow  \phi(a) = \phi(b) \rightarrow b \sim a$。

        (3) 传递性:$a\sim b, b \sim c \rightarrow \phi(a) = \phi(b) = \phi(c) \rightarrow a \sim c$。

        令$[a] = \{x | \phi(x) = \phi(a)\}$,全体等价类为$\{[a] | a \in A \}$

    \end{solution}    
\end{problem} 

\begin{problem} Page 6-8
    \begin{solution}
        (1) 反身性: $ab = ab \rightarrow (a, b) \sim (a, b) $。

        (2) 对称性:$(a, b) \sim (c, d) \rightarrow ad = bc \rightarrow (c, d) \sim (a, b)$。

        (3) 传递性:$(a, b) \sim (c, d), (c, d) \sim (e, f) \rightarrow ad = bc, cf = de$。
        两式两边相乘消去 $dc$ 得 $ af=be \rightarrow (a, b) \sim (e, f)$。得证。
    \end{solution}
\end{problem} 

\begin{problem} Page 16-5
    \begin{solution}
        (1) 封闭性: $a \oplus b = a + b - 2 \in \mathbb{Z} $。

        (2) 结合律: 
            \begin{align}
                (a \oplus b) \oplus c &= (a + b - 2) + c - 2   \\
                    &= a + (b + c - 2) - 2  \\
                    &= a \oplus (b \oplus c)
            \end{align}
        
        (3) 存在零元:
            $ a \oplus 2 = a + 2 - 2 = a$。
        
        (4) 存在负元:
            $ a \oplus 4 - a = a + 4 - a - 2 = 2$。
            

    \end{solution}
\end{problem} 
\begin{problem} Page 17-12
    \begin{solution}
        任取 $x, y \in G$, 由题意,有 $(xy)^2 = e = (yx)^2$。 
        
        又有$\forall a \in G$, $a=a^{-1}$
        
        整理得:
        $(xy)^2=(yx)^2=(yx)^{-1}yx=xy\cdot yx \Rightarrow xy=yx$。
    
        所以$G$是一个交换群。

    \end{solution}
\end{problem}

\begin{problem} Page 17-13
    \begin{solution}
        \textbf{必要性}:由交换群的性质有,$(ab)^2=a(ba)b=a(ab)b=a^2b^2$。
        \textbf{充分性}:$\forall a, b \in G$, $(ab)^2=a^2b^2 \Rightarrow abab=aabb \Rightarrow ab=ba$。
        所以$G$为阿贝尔群。

    \end{solution}
\end{problem}

\begin{problem} Page 17-16
    \begin{solution}
        令 $S=\{ x | x \in G, x^3 = e\}$。
        
        显然 $e \in S$, 若 $\exists a \in S$, $a \not= e$, 则$a^2 \in S$. 且根据假设$a \not= e$, 所以 $a \not= a^2$。
        且对于 $\forall a, b \in S, a \not= b, a \not= b^2$, 若 $a^2=b^2$, 则$a^3=e=ab^2\rightarrow b = a$, 产生矛盾。
        所以对于$S$中除$e$外的元素都成对存在, 又$G$为有限群,所以元素个数为奇数。        
    \end{solution}
\end{problem}


\end{document}