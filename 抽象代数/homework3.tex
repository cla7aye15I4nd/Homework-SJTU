\documentclass{ximera}
\usepackage[UTF8]{ctex}

\title{MA150 Algebra }
\author{于峥}

\begin{document}

\begin{abstract}
    homework 3
\end{abstract}
\maketitle

\begin{problem} Page 32-3
    \begin{solution} \textbf{必要性}:如果 $G$ 是交换群,
        \begin{enumerate}
            \item 单射性质: $\forall x, y \in G$, 若$\phi(x) = \phi(y)$, 即 $x^{-1}=y^{-1}$。 两边乘$xy$得:
            
            $(xy)x^{-1}=(xy)y^{-1} \Rightarrow (yx)x^{-1}=y(xx^{-1})=x(yy^{-1})\Rightarrow y=x$
            \item 满射性质:$\forall x \in G^{'}$,根据定义$x^{-1} \in G$。
            \item 保持运算: $\forall x, y \in G$, $\phi(xy)=(xy)^{-1}=y^{-1}x^{-1}=x^{-1}y^{-1}=\phi(x)\phi(y)$
        \end{enumerate}
        
        \textbf{充分性}:因为 $\phi(x)=x^{-1}$是 $G \rightarrow G^{'}$的同构映射:

            $\forall x, y \in G$, 有$\phi(xy)=(xy)^{-1}=y^{-1}x^{-1}$, 又$\phi(xy)=\phi(x)\phi(y)=x^{-1}y^{-1}$。
            
            所以$xy=yx$, $G$是交换群。
    \end{solution}
\end{problem}

\begin{problem} Page 32-4
    \begin{solution} 分三步证明证明
        \begin{enumerate} 
            \item 单射性质: $\forall x, y \in G$, 若$\phi(x) = \phi(y)$, 即 $axa^{-1}=aya^{-1}$。 两边左乘$a^{-1}$,右乘$a$得$x=y$。
            \item 满射性质:$\forall axa^{-1} \in G$,因为$a \in G$, 所以$x \in G$。
            \item 保持运算: $\forall x, y \in G$, $\phi(xy)=axya^{-1}=axa^{-1}aya^{-1}=\phi(x)\phi(y)$。
        \end{enumerate}
    \end{solution}
\end{problem}

\begin{problem} Page 33-6
    \begin{solution}
        $G = (\mathbb{R}, +)$,$H = (4\mathbb{R}, +)$
    \end{solution}
\end{problem}

\begin{problem} Page 41-1
    \begin{solution} $Z_n$中,$\text{ord}(m)=\frac n {(n,m)}$
        \begin{enumerate}[label=(\arabic*)]
            \item
            \begin{tabular}{|c|c|c|c|c|c|c|c|}
                \hline n&0&1&2&3&4&5&6\\
                \hline ord(n)&1&7&7&7&7&7&7\\
                \hline
            \end{tabular}
            \item
            \begin{tabular}{|c|c|c|c|c|c|c|c|c|}
                \hline n&0&1&2&3&4&5&6&7\\
                \hline ord(n)&1&8&4&8&2&8&4&8\\
                \hline
            \end{tabular}
            \item
            \begin{tabular}{|c|c|c|c|c|c|c|c|c|c|c|}
                \hline n&0&1&2&3&4&5&6&7&8&9\\
                \hline ord(n)&1&10&5&10&5&2&5&10&5&10\\
                \hline
            \end{tabular}
            \item
            \begin{tabular}{|c|c|c|c|c|c|c|c|c|c|c|c|c|c|c|}
                \hline n&0&1&2&3&4&5&6&7&8&9&10&11&12&13\\
                \hline ord(n)&1&14&7&14&7&14&7&2&7&14&7&14&7&14\\
                \hline
            \end{tabular}
            \item
            \begin{tabular}{|c|c|c|c|c|c|c|c|c|c|c|c|c|c|c|c|}
                \hline n&0&1&2&3&4&5&6&7&8&9&10&11&12&13&14\\
                \hline ord(n)&1&15&15&5&15&3&5&15&15&5&3&15&5&15&15\\
                \hline
            \end{tabular}
            \item
            \begin{tabular}{|c|c|c|c|c|c|c|c|c|c|c|c|c|c|c|c|c|c|c|}
                \hline n&0&1&2&3&4&5&6&7&8&9&10&11&12&13&14&15&16&17\\
                \hline ord(n)&1&18&9&6&9&18&3&18&9&2&9&18&3&18&9&6&9&18\\
                \hline
            \end{tabular}            
        \end{enumerate}
        
    \end{solution}
\end{problem}

\newpage
\begin{problem} Page 42-5
    \begin{solution} 得到答案需要两步:
        \begin{itemize}
            \item 判断$U(n)$是否有生成元:
    
            形如 $2, 4, p^n, 2p^n$的数有生成元,其中$p$是奇素数。
            \item 寻找所有的生成元没有什么好方法,只能暴力枚举。
        \end{itemize}
        \begin{tabular} {|c|c|c|c|}
            \hline 9&10&13&14 \\
            \hline 2, 5&3, 7&2, 6, 7, 11&3, 5 \\
            \hline
        \end{tabular}
    \end{solution}
\end{problem}

\begin{problem} Page 42-12
    \begin{solution} 不妨设$\text{ord}(a)=r$, 即$a^r=e$。

        (1) $(gag^{-1})^r=ga^rg^{-1}=gg^{-1}=e$。

        (2) 假设存在$s<r$,$(gag^{-1})^s=ga^sg^{-1}=e$。

            则 $ga^s=g \Rightarrow a^s=e$, 与前提矛盾。

        所以$gag^{-1}$的阶与$a$相同。
    \end{solution}
\end{problem}

\end{document}