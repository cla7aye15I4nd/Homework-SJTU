\documentclass{ximera}
\usepackage[UTF8]{ctex}

\title{MA150 Algebra}
\author{于峥}

\begin{document}
\begin{abstract}
    homework 6
\end{abstract}

\maketitle

\begin{problem} Page 79-2
    \begin{solution}
        根据$C(G)$的定义
        $$C(G)=\{g \in G | gx=xg, \forall x \in G\}$$
        则 $\forall a \in G$
        $$
        \begin{aligned}           
            aC(G)a^{-1} =& \{axa^{-1} | x \in C(G) \}  \\
            =& \{ x(aa^{-1}) | x \in C(G)\}  \\
            =& C(G) \\
        \end{aligned}
        $$
        所以$C(G)$是$G$的正规子群。
    \end{solution}
\end{problem}

\begin{problem} Page 79-5
    \begin{solution}
        $$
        \begin{aligned}
            H =& \{(1), (2\ 4)\} \\
            K =& \{(1), (2\ 4), (1\ 3), (2\ 4)(1\ 3)\} \\
            G =& \{(1), (2\ 4), (1\ 3), (2\ 4)(1\ 3), (1\ 2)(3\ 4), (2\ 3)(1\ 4), (1\ 2\ 3\ 4), (1\ 4\ 2\ 3)\}
        \end{aligned}
        $$
        其中
        $(1\ 2\ 3\ 4)(1\ 3)(1\ 2\ 3\ 4)^{-1} = (2\ 4) \not\in H$
    \end{solution}
\end{problem}

\newpage
\begin{problem} Page 79-6
    \begin{solution}
        \textbf{必要性}:若$H$是$G$d的正规子群,现有$a,b \in G$, $ab \in H$。
        
        因为$bHb^{-1} = H$, 所以$b(ab)b^{-1} = ba \in H$。

        \textbf{充分性}:若$ab \in H$, 都有$ba \in H$,则$a(a^{-1}h) \in H \Rightarrow (a^{-1}h)a \in H$。所以$H$是正规子群。
    \end{solution}
\end{problem}

\begin{problem} Page 79-9
    \begin{solution}
        根据定义, $\forall a \in N(H)$, $aHa^{-1}=H$ ,$H$是$N(H)$的正规子群是显然的。

        首先$N(H) \subseteq G$, $\forall a, b \in N(G)$。
        $$
        \begin{aligned}           
            (ab^{-1})H(ab^{-1})^{-1} &=a(b^{-1}Hb)a^{-1} \\
            &=a(b^{-1}(bHb^{-1})b)a^{-1} \\
            &=aHa^{-1} \\
            &=H
        \end{aligned}
        $$
        所以$ab^{-1} \in N(H)$, $N(H)$是$G$的子群。
    \end{solution}
\end{problem}

\begin{problem} Page 79-10
    \begin{solution}
        \textbf{必要性}:由于$H$是正规子群,所以$\forall a, aHa^{-1}=H \Rightarrow \phi(H) \in H$。

        \textbf{充分性}:由于$\phi(H) \in H$, 即$\forall a$, $aHa^{-1} \in H$, 所以$H$是$G$的正规子群。
    \end{solution}
\end{problem}

\begin{problem} Page 79-11
    \begin{solution}
        考虑商群$G / H$, $H$为商群的单位元, 由于$|G / H| = [G : H] = m$。根据拉格朗日定理的推论,$(xH)^m=x^mH=\overline e = H$。
        所以$x^m \in H$
    \end{solution}
\end{problem}

\end{document}