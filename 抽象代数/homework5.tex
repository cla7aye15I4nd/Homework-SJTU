\documentclass{ximera}
\usepackage[UTF8]{ctex}

\title{MA150 Algebra}
\author{于峥}

\begin{document}

\begin{abstract}
    homework 3
\end{abstract}

\begin{problem} Page 71-1
    \begin{solution}
        $A_4$中所有的左陪集有
        \begin{itemize}
            \item $(1)H = \{(1), (1~2)(3~4),(1~3)(2~4),(1~4)(2~3)\}$
            \item $(1~2~3)H=\{(1~2~3),(1~3~4),(2~4~3),(1~4~2)\}$
            \item $(1~3~2)H=\{(1~3~2),(1~4~3),(2~3~4),(1~2~4)\}$
        \end{itemize}
        $S_4$中所有的左陪集有
        \begin{itemize}
            \item $(1)H = \{(1), (1~2)(3~4),(1~3)(2~4),(1~4)(2~3)\}$
            \item $(1~2~3)H=\{(1~2~3),(1~3~4),(2~4~3),(1~4~2)\}$
            \item $(1~3~2)H=\{(1~3~2),(1~4~3),(2~3~4),(1~2~4)\}$
            \item $(1~2)H=\{(1~2),(3~4),(1\ 3\ 2\ 4),(1\ 4\ 2\ 3)\}$
            \item $(1~3)H=\{(1~3),(2~4),(1\ 2\ 3\ 4),(1\ 4\ 3\ 2)\}$
            \item $(2~3)H=\{(1~4),(2~3),(1\ 2\ 4\ 3),(1\ 3\ 4\ 2)\}$
        \end{itemize}

    \end{solution}
\end{problem}

\begin{problem} Page 71-8
    \begin{solution}
        因为 $ord(a) = 30$, 所以$a ^ {30} = e$。$ord(a^4) = \frac {30} {gcd(30, 4)} = 15$。
        所以有$2$个左陪集。
        分别为$a\langle a^4 \rangle=\{a^i|i = 2k+1, k\in \mathbb{N}\}$,
        $e\langle a^4 \rangle=\{a^i|i = 2k, k\in \mathbb{N}\}$。
    \end{solution}
\end{problem}

\begin{problem} Page 71-11
    \begin{solution}
        设$H$是群$G$的子群,$aH$是$H$的左陪集。
        \begin{itemize}
        \item 若$a \in H$,  则$aH=H=H^{-1}=Ha$。
        \item 若$a \not\in H$, 则$\forall h \in H, (ah)^{-1} = h^{-1}a^{-1}$。
        所以$(aH)^{-1}=H^{-1}a^{-1}=Ha^{-1}$。
        \end{itemize}
    \end{solution}
\end{problem}

\begin{problem} Page 71-12
    \begin{solution}
        \begin{itemize}
            \item $a(H_1 \cap H_2) \subseteq aH_1 \cap aH_2$:
            
            $\forall h \in H_1 \cap H_2$, $ah \in a(H_1 \cap H_2)$, 又$ah \in aH_1m ah \in aH_2$。 所以

            $ah \in aH_1 \cap aH_2 \Rightarrow a(H_1 \cap H_2) \subseteq aH_1 \cap aH_2$。
            \item $aH_1 \cap aH_2 \subseteq a(H_1 \cap H_2)$:
                
            $\forall h_1 \in H_1 / H_2, h_2 \in H_2 / H_1$,显然$ah_1 \not= ah_2$, 而

            $\forall h \in H_1 \cap H_2$, 上面已经证明。所以$aH_1 \cap aH_2 \subseteq a(H_1 \cap H_2)$。
        \end{itemize}
        所以 $a(H_1 \cap H_2) = aH_1 \cap aH_2$。
    \end{solution}
\end{problem}
 
\begin{problem} Page 72-20
    \begin{solution}
        \textbf{方法一}:
        假设没有阶为$3$的元素。

        首先,若$\exists a, ord(a) = 33$, 则$ord(a^{11}) = 3$。

        所以假设只存在阶为$11$的元素$a$,令$A=\{a^k|k \in \mathbb{N}\}$。取$b$为$H$外$G$中某个元素。 根据假设只能有$ord(b)=11$, 令$B=\{b^k|k \in \mathbb{N}\}$。
        则$AB \subseteq G$和所以存在$a^{x_0}b^{y_0}=a^{x_1}b^{y_1}$, 即存在$a^m=b^n$。而$11$是素数,所以$\langle a \rangle = \langle a^n \rangle = \langle b^m \rangle = \langle b \rangle$,产生矛盾!
        所以$ord(b)$只能为$3$, 所以必定存在$3$阶元素。

        \textbf{方法二}:
        当$A,B$为$G$的有限子群时, 运用$|AB||A\cap B|=|A|\cdot|B|$可以很方便的证明:
        $|AB| = \frac {|A||B|} {|A \cap B|} = 121 > |G|$。矛盾,所以必有$ord(b)=3$。
        


    \end{solution}
\end{problem}

\begin{problem} Page 72-22
    \begin{solution} 设映射$\phi(a) = a^n$ 为$G \rightarrow G$ 的映射。 
        \begin{itemize}
            \item \textbf{封闭性}:$\forall a, b \in G(a \not= b)$, $\phi(ab)=(ab)^n=a^nb^n=\phi(a)\phi(b)$。
            \item \textbf{单映射}:$\forall a, b \in G$, 若$a^n=b^n$, 则$(ab^{-1})^n=e$, 由于$n$与$|G|$互素,所以$ab^{-1}=e$,所以$a=b$。
            \item \textbf{满映射}:由于$\phi$为单映射,所以$|G|=|\phi(G)|$,所以$\phi$为满映射。
        \end{itemize}
        所以$\phi$为$G \rightarrow G$自同构。
    \end{solution}
\end{problem}

\end{document}