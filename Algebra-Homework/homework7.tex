\documentclass{ximera}
\usepackage[UTF8]{ctex}

\title{MA150 Algebra}
\author{于峥}

\begin{document}
\begin{abstract}
    homework 7
\end{abstract}

\maketitle

\begin{problem} Page 88-1
    \begin{solution}
        \begin{itemize}
            \item $\phi(x) = |x|$
            $\forall x, y \in R^*$, $\phi(xy)=|xy|=|x|\cdot|y|=\phi(x)\phi(y)$。所以$\phi$是同态映射。
            $Ker \phi = \{1, -1\}$,$\phi(G)=R^+$。
            \item $\phi(x) = ax$, 
            $\forall x, y \in R^*$, $\phi(xy)=axy$。若$a = 1$,则$\phi$是同态映射,$Ker \phi = \{1\}$,$\phi(G)=G$。否则不是。
            \item $\phi(x) = x^2$,
            $\forall x, y \in R^*$, $\phi(xy) = (xy)^2 = x^2y^2 = \phi(x)\phi(y)$,则$\phi$是同态映射。
            $Ker \phi = \{1, -1\}, \phi(G)=R^+$
            \item $\phi(x) = - \frac 1 x$,
            $\forall x, y \in R^*$, $\phi(xy) = - \frac 1 {xy} \not= \phi(x)\phi(y)$,则$\phi$不是同态映射。
        \end{itemize}
    \end{solution}
\end{problem}

\begin{problem} Page 88-6
    $$\phi(a + bi) = (a + bi)^6$$
    则$\forall x, y \in \mathbb{C}^*$, $x = a + bi$, $y = c + di$。
    $$
        \begin{aligned}
            \phi(xy) &= (xy)^6 \\
            &= x^6y^6 \\
            &= \phi(x)\phi(y)
        \end{aligned}
    $$
    $Ker(\phi) = \{ x \in C^* | x^6 = 1 \} = \bigcup_{x=0}^5\{ \cos(\frac x 3 \pi) + i\sin(\frac x 3 \pi) \}$
\end{problem}

\begin{problem} Page 88-7
    \begin{solution}
        设$\phi$为$\mathbb{Z} \rightarrow \mathbb{Z}_m$的同态映射,令$\phi(x)=\overline{x^{'}}$。

        则$\phi(x+y)=\phi(x)+\phi(y)=\overline{x^{'}}+\overline{y^{'}}=\overline{(x^{'}+y^{'}}) \Rightarrow \forall a \in \mathbb{Z}$,$\phi(ax)=a\overline{x^{'}}$
        
        所以$\phi(xy)=x\overline{y^{'}}=y\overline{x^{'}} \Rightarrow xy^{'} \equiv yx^{'} (\mod m)$。

        由于$x, y$的任意性,即有可能$\gcd(x,y)=1$,所以必有$x^{'}=xa$,$y^{'}=ya$。

        推出$\phi(x)=x\overline{a}(a = 0, 1, \dots, m-1)$共$m$个映射满足要求。
    \end{solution}
\end{problem}

\begin{problem} Page 89-16
    \begin{solution}
        \textbf{充分性}:若$aKer \phi = b Ker\phi$, 
        则$\phi(aKer \phi) = \{ x \in G^{'} | x = \phi(a) \} = \phi(bKer \phi) = \{ x \in G^{'} | x = \phi(b) \}$。
        所以$\phi(a) = \phi(b)$。

        \textbf{必要性}:若$\phi(a)=\phi(b)$, $\forall x \in aKer \phi$, $x = at(t \in Ker \phi)$, 
        
        又$\phi(b^{-1}al)=\phi(b^{-1})\phi(a)=\phi(b^{-1})\phi(b)=\phi(e)=e^{'}$。
        
        所以$b^{-1}al \in Ker \phi \Rightarrow x \in bKer \phi$。

        即$\forall x \in a Ker \phi, x \in bKer \phi$,$aKer \phi = bKer \phi$。
    \end{solution}
\end{problem}

\newpage
\begin{problem} Page 89-18
    \begin{solution}
        \begin{itemize}
            \item $\forall a \in HK$,$\exists h \in H$,$k \in K$,$a = hk$。
            
            $\phi(a)=\phi(h)\phi(k)=\phi(h) \in \phi(H)$, 则$a \in \phi^{-1}(\phi(H))$。

            \item $\forall a \in \phi^{-1}(\phi(H))$, 则$\phi(a) \in \phi(H)$, 
            
            于是$\exists h \in H$, $\phi(h)=\phi(a) \Rightarrow \phi(h^{-1})\phi(a)=\phi(h^{-1})\phi(h)=e^{'}$, 所以$h^{-1}a \in K$。
            所以$h\cdot h^{-1}a \in HK$。

            综上,$\phi^{-1}(\phi(H)) = HK$。
        \end{itemize}
    \end{solution}
\end{problem} 

\begin{problem} Page 89-19
    \begin{solution}
        根据题意, $G_1 = \langle a \rangle$,$G_2 = \langle b\rangle$。

        $\Leftarrow$: (1) 令$\phi(a^x) = b^x$, 显然这是一个从 $G_1  \rightarrow G_2$ 的映射,
        
        (2) 又$\phi(a^{xy})=b^x=b^x\cdot b^y=\phi(a^x)\phi(a^y)$,所以这是一个同态映射。
        
        (3) $n_2|n_1 \Rightarrow n_1 \leq n_2$, 所以$\forall b^x \in G_2, \exists a^x \in G_1$, 使得$\phi(a^x)=b^x$,所以$\phi$是一个满同态。

        $\Rightarrow$: 因为$G_1 \sim G_2$, 所以$G_1 / Ker \phi \cong G_2$, 所以$\frac {n_1} {n_2} = |Ker \phi| \Rightarrow n_2 | n_1$。
    \end{solution}    
\end{problem} 

\end{document}