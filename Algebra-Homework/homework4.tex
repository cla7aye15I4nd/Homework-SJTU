\documentclass{ximera}
\usepackage[UTF8]{ctex}

\title{MA150 Algebra}
\author{于峥}

\begin{document}

\begin{abstract}
    homework 3
\end{abstract}
\maketitle

\begin{problem} Page 54-5
    \begin{solution}
        $$    
            \tau\sigma\tau^{-1}=
            \left\lgroup
            \begin{aligned}
                \tau(1)~~&~~~~~\tau(2)~~\dots~~~~~~\tau(n) \\
                \tau(\sigma(1))~~&\tau(\sigma(2))~\dots~~\tau(\sigma(n)) \\
            \end{aligned}
            \right\rgroup
        $$
        
        令$\sigma=(k_1,k_2,\dots,k_3)$

        则 $\tau\sigma\tau^{-1}= (\tau(k_1),\tau(k_2)\dots,\tau(k_n))$

        (1) (3 1 4 2)

        (2) (2 4)(3 1)
    \end{solution}
\end{problem}

\begin{problem} Page 55-12
    \begin{solution} 共有6个
        \begin{itemize}
            \item $\{(1)\}$
            \item $S_3$
            \item $\{(1), (1,2)\}$
            \item $\{(1), (1,3)\}$
            \item $\{(1), (2,3)\}$
            \item $\{(1), (1,2,3),(3,2,1)\}$
        \end{itemize}
    \end{solution}
\end{problem}

\begin{problem} Page 55-24
    \begin{solution}
        首先由于$G$是非空置换群,所以单位元$(1) \in G$。
        
        所以根据题设,$G$中既有奇置换又有偶置换。设奇偶置换的集合为$A, B$。

        根据奇偶置换的性质,$\forall \sigma \in A$, 有 $\sigma A \subseteq B, \sigma B \subseteq A$。

        则$|\sigma A| = |A| \leq |B|$, $|\sigma B| = |B| \leq |A| \Rightarrow |A| = |B|$。
    \end{solution}
\end{problem}

\begin{problem} Page 55-25
    \begin{solution}
        首先$(1)$是偶置换,又偶置换的逆为偶置换,偶置换的乘积为偶置换,所以偶置换集合是一个子群。
    \end{solution}
\end{problem}

\end{document}