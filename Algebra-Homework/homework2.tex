\documentclass{ximera}
\usepackage[UTF8]{ctex}

\title{MA150 Algebra }
\author{于峥}

\begin{document}

\begin{abstract}
    homework 2
\end{abstract}
\maketitle

\begin{problem} Page 25-5
    \begin{solution}
        首先根据题设,显然有 $H \subseteq G$。

        对$\forall a, b \in H$, 因$G$是交换群,则有
        $(ab^{-1})^m=a^m\cdot (b^{-1})^m=e\rightarrow ab^{-1}\in H$。

        所以$H$是$G$的子群。
    \end{solution}
\end{problem}

\begin{problem} Page 25-6
    \begin{solution}
        $gHg^{-1} \subseteq G, \forall a, b \in H, gag^{-1}, gbg^{-1} \in gHg^{-1}$。
        \begin{equation}
            gag^{-1}\cdot (gbg^{-1})^{-1}=gag^{-1}\cdot gb^{-1}g^{-1}=g(ab^{-1})g^{-1}
        \end{equation}
        $H < G \rightarrow ab^{-1} \in H \rightarrow g(ab^{-1}) g^{-1} \in gHg^{-1} \rightarrow gHg^{-1} < G$。
    \end{solution}
\end{problem}

\begin{problem} Page 25-7
    \begin{solution}
        显然$C(a) \subseteq G$,

        (1) $\forall g \in C(a), ga = ag$, 等式两边变换后得到
        
        $g^{-1}(ga)g^{-1}=g^{-1}(ag)g^{-1} \rightarrow ag^{-1}=g^{-1}a$。

        所以$g^{-1} \in C(a)$。

        (2) $\forall x, y \in C(a)$,
        
        $xya=x(ya)=xay=axy \rightarrow xy \in C(a)$。

        综上 $C(a) < G$。

    \end{solution}
\end{problem}

\begin{problem} Page 25-8
    \begin{solution}
        此题分两步分别证明:
        
        (1) $C(G) \subseteq \bigcap_{a\in G} C(a)$

        $\forall x \in C(G)$, 根据$C(G)$的定义,

        有$\forall a \in G, x \in C(a)$, 所以$a \in \bigcap_{a\in G} C(a)$,结论得证。    

        (2) $C(G) \supseteq \bigcap_{a\in G} C(a)$。

        $\forall x \in \bigcap_{a\in G} C(a)$,则$\forall g \in G, gx=xg$,
        所以$x \in C(g)$。

        综上 $C(G) = \bigcap_{a\in G} C(a)$。
    \end{solution}
\end{problem}

\begin{problem} Page 26-18
    \begin{solution}
        首先 $\left \langle m, n \right \rangle = \{am+bn|a,b\in \mathbb{Z}\}$, $\left \langle d \right \rangle = \{kd|k\in \mathbb{Z}\}$。

        下证$ax+by=m$有解当且仅当$m$为$d=\gcd(a,b)$的整数倍。

        (1) 如果有$a=0$或$b=0$, 则显然成立。

        (2) 若$a,b$都不为$0$, $\forall x, y \in \mathbb{Z}$, 有 $d|(ax+by)$。
            
            设$s>0$为$ax+by$的最小值,因为$d|(ax+by)$,则必有$d|s$。

            令 $q = \left \lfloor \frac a s \right \rfloor$,$r = a \mod s$。
            
            容易发现 $r = a - q(ax + by) = a(1-qx)+b(-qy)$。

            又$s$为最小值,所以$r=0$。得到$s | a$的结论。

            同理得到$s | b$, 那么$s|d$所以有$s = d$。

            所以$ax+by=d$存在解,进而$ax+by=kd, k \in Z$有解。

        综上所述,$\forall x \in \left \langle m, n \right \rangle, x \in \left \langle d \right \rangle$。
        反之亦然。得出结论$\left \langle m, n \right \rangle = \left \langle d \right \rangle$。

            
    \end{solution}
\end{problem}

\begin{problem} Page 26-19 
    \begin{solution}
        \textbf{必要性}:当$m=\pm n$,显然有 $\left \langle m \right \rangle  = \left \langle n \right \rangle$。
        
        \textbf{充分性}: 反证法,若$m \not= \pm n$,且$\left \langle m \right \rangle  = \left \langle n \right \rangle$。

        令$m=pn+r$, 因为$\left \langle m \right \rangle = \{km|k \in \mathbb{Z}\}$。
        
        因为 $kn \in \left \langle m \right \rangle$, 所以 $kpn+kr=kn$均有解。那么$r=0$。

        因为 $n \in \left \langle m \right \rangle$, 所以 $kp=1$有解。那么$p=\pm 1$。

        这与假设矛盾。

        综上所述,$\left \langle m \right \rangle  = \left \langle n \right \rangle$ 当且仅当 $m=\pm n$。
    \end{solution}
\end{problem}

\begin{problem} \textbf{证明}:若$N = n_1n_2,\gcd(n_1,n_2)=1$
    那么 $\mathbb{Z}_N^*\cong\mathbb{Z}_{n_1}^*\times\mathbb{Z}_{n_2}^*$。
    \begin{solution}
        (1) 构造$\mathbb{Z}_N^*$到$\mathbb{Z}_{n_1}^*\times\mathbb{Z}_{n_2}^*$的映射$\phi(x)$。

        $$\phi(x) = (x \mod n_1, x \mod n_2)$$

        (2) \textbf{双射性质}:$\forall x, y \in \mathbb{Z}_{N}$, 使得
        $\phi(x)=\phi(y)$。若$x\not=y$, 即
        
        $$ 
        \left\{
            \begin{aligned}
                x \mod n_1 &= a \\
                x \mod n_2 &= b
            \end{aligned}
        \right.
        $$

        有至少两个解,然而$\gcd(n_1,n_2)=1$, $0 < x, y < N$。根据中国剩余定理,矛盾。所以$\phi(x)$为单射。
        同时由于该方程必有解,所以$\phi(x)$为满射。

        (3) \textbf{保持运算}:
            $\forall x, y \in \mathbb{Z}_N^*, \phi(x) = (a, b), \phi(y) = (c, d)$。

            $$
                \begin{aligned}                    
                    \phi(xy)&=(xy \mod n_1, xy \mod n_2)\\
                    &=(ac \mod n_1, bd \mod n_2)\\
                    &=\phi(x)\phi(y)
                \end{aligned}.
            $$。
        综上所述,得证。
    \end{solution}
\end{problem}

\end{document}