\documentclass[12pt,a4]{article}
\usepackage{graphicx,amsmath,amssymb,amsthm, boxedminipage}
\usepackage{algorithm}
\usepackage{algpseudocode}


\newtheorem{theorem}{Theorem}%[section]
\newtheorem{proposition}[theorem]{Proposition}
\newtheorem{lemma}[theorem]{Lemma}
\newtheorem{corollary}[theorem]{Corollary}
\newtheorem{definition}[theorem]{Definition}

\newcommand{\scalar}[2]{\ensuremath{\langle #1, #2\rangle}}
\newcommand{\floor}[1]{\left\lfloor #1 \right\rfloor}
\newcommand{\ceil}[1]{\left\lceil #1 \right\rceil}
\newcommand{\norm}[1]{\|#1\|}
\newcommand{\pfrac}[2]{\left(\frac{#1}{#2}\right)}
\newcommand{\nth}[1]{#1\textsuperscript{th}}

% \newcommand{\nth}[1]{#1\textsuperscript{th}}
\newcommand{\E}{\mathop{\mathbb{E\/}}}
\newcommand{\N}{\mathbb{N}}
\newcommand{\R}{\mathbb{R}}

\newtheorem{exercise}[theorem]{Exercise}
\newtheorem{exerciseD}[theorem]{*Exercise}
\newtheorem{exerciseDD}[theorem]{**Exercise}

\let\oldexercise\exercise
\renewcommand{\exercise}{\oldexercise\normalfont}

\let\oldexerciseD\exerciseD
\renewcommand{\exerciseD}{\oldexerciseD\normalfont}

\let\oldexerciseDD\exerciseDD
\renewcommand{\exerciseDD}{\oldexerciseDD\normalfont}


 
\begin{document}

\author{NoCode}
\date{\today}

\title{CS 217 -- Algorithm Design and Analysis \\ 
  \vspace{3mm}
{\large	Shanghai Jiaotong University, Fall 2019\\
}
}
\maketitle



\setcounter{section}{1}


\section{Sorting Algorithms}

\begin{exercise}
   Given an array $A$ of $n$ items (numbers), we can find the maximum with $n-1$ comparisons (this is simple).
   Show that this is optimal: that is, any algorithm that does $n-2$ or fewer comparisons will fail to find the maximum 
   on some inputs.
   
   \paragraph{Solution}
  A number is known as the maximum only if everyone else is knwon smaller than that number directly or indirectly. If only $n-2$ or fewer comparisons taken, there must be at least two numbers that are never compared directly or indirectly, which means that is impossible to determine the maximum.
\end{exercise}

\begin{exercise}
  Let $A$ be an array of size $n$, where $n$ is even. 
  Describe how to find both the minimum and the maximum
  with at most $\frac{3}{2} n  - 2$ comparisons.
  Make sure your solution is {\em simple}, in describe it 
  in a clear and succinct way!
  \paragraph{Solution}
    Suppose the $A = \{a_1, a_2, \cdots, a_n\}$, first we divide $A$ in $\frac n 2$ group $\{(a_1,a_2), (a_3, a_4), \cdots,$
    $ (a_{n-1}, a_{n})\}$, then we
    compare each group, put each larger element in $B$, and put each smaller element in $C$, this take $\frac n 2$ comparisons. Then we compare every two adjacent elements 
    in $B$ and $C$,then we can get the maximum element in $B$ and the minimum element in $C$, which takes $(\frac n 2 -1)*2$ comparisons. Because 
    the minimum element in $A$ is the minimum element in $C$, the maximum element in $A$, so we find the maximumand the minimum with at most $\frac {3n}{2} -2$ comparisons
\end{exercise}

\begin{exercise}
  Given an array $A$ of size $n = 2^k$, find the second largest element element
  with at most $n + \log_2(n)$ comparisons. 
  Again, your solution should be {\em simple}, and you should explain
  it in a clear and succinct way!
  
  \paragraph{Solution}
First we divide $A$ in $2^{k-1}$ group $\{(a_1,a_2), (a_3, a_4), \cdots,$
$ (a_{n-1}, a_{n})\}$, we compare each group, put each larger element in $A_1$. Then we continuely divide $A_1$ in $2^{k-2}$ group, compare each group and put each larger element in $A_2$.
We repeatedly do this until we find the maximum element $M$, which totally takes $n$ comparisons. In this process, we have $log_2 n = k$ elements compared with $M$ in some time. We only need to 
find the maximum in these $k$ elements, which is the second largest element in $A$. This take $k-1$ comparisons. So we totally take at most $n+log_2 n$ comparisons to find the second largest element.
\end{exercise}


Recall the quicksort tree defined in the lecture.
 \begin{center}
 	\includegraphics[width=\textwidth]{figures/quicksort-tree.pdf}
\end{center}

We denote a specific list (ordering) by $\pi$ and the tree by $T(\pi)$. $A_{i,j}$ is an indicator
variable which is $1$ if $i$ is an ancestor of $j$ in the tree $T(\pi)$, and $0$ otherwise.
In the lecture, we have derived:
\begin{align*}
	\E[A_{i,j}] & = \frac{1}{|i-j|+1} \\
	\textnormal{ total number of comparisons } & = \sum_{i \ne j} A_{i,j} \ . 
\end{align*}


\begin{exercise}
 Determine the expected number of comparisons made by quicksort. 
 Your final formula must be {\em closed}, meaning it must not contain 
 $\E$, $\prod$, or $\sum$. It may, however, contain $H_n := 1 + \frac{1}{2} + \frac{1}{3} + 
 \dots + \frac{1}{n}$ , the 
 $\nth{n}$ Harmonic number. \textbf{Remark.} This gets a bit tricky, and you will need some
 summation wizardry towards the end.
 \end{exercise}
 

\subsection{Quickselect}

Remember the recursive algorithm \textsc{QuickSelect} from the lecture. I write
it below in pseudocode. In analogy to quicksort we define QuickSelect deterministically
and assume that the input array is random, or has been randomly shuffled before
QuickSelect is called. We assume that $A$ consists of distinct elements and
$1 \leq k \leq |A|$.

\begin{algorithm}
\caption{Select the \nth{$k$} smallest element from a list $A$}
\begin{algorithmic}[1]
\Procedure{QuickSelect}{$X,k$}
  \If{$|X| = 1$}
    \State \Return $X[1]$
  \Else:
  \State $p := X[1]$
  \State $Y:= [ x \in X \ | \ x < p]$
  \State $Z := [ x \in X \ | \ x > p]$
  \If{$|B| = k-1$}
   	\State \Return $p$
  \ElsIf{$|Y| \geq k$}
     	\State \Return $\textsc{QuickSelect}(Y,k)$
  \Else
  	\State Return $\textsc{QuickSelect}(Z, k- |Y| - 1)$
  \EndIf
  \EndIf
\EndProcedure
\end{algorithmic}
\end{algorithm}


Let $C$ be the number of comparison made by $\textsc{QuickSelect}$. In the 
lecture we proved that $\E[C] \leq O(n)$ when we run it on a random input.

\begin{exercise}
	Explain how \textsc{QuickSelect} can be viewed as a ``partial execution'' of quicksort
	with the random pivot selection rule.
	Draw an example quicksort tree and show which part of this tree is visited
	by \text{QuickSelect}.
\end{exercise}

Let $B_{i,j,k}$ be an indicator variable which is $1$ if $i$ is a common ancestor
of $j$ and $k$ in the quicksort tree. That is, if both $j$ and $k$ appear in the 
subtree of $T(\pi)$ rooted at $i$.

\begin{exercise}
   What is $\E[B_{i,j,k}]$? Give a succinct formula for this.
   
   \paragraph{Solution}
   $$
   \E[B_{i,j,k}]= \frac 1 {\max(i,j,k)-\min(i,j,k)+1}
   $$
\end{exercise}



\begin{exercise}
  Let $C(\pi,k)$ be the number of comparisons made by \textsc{QuickSelect} when given
  $\pi$ as input. Design a formula for $C(\pi,k)$ with the help of the indicator
  variables $A_{i,j}$ and $B_{i,j,k}$ (analogous to the formula 
  $\sum_{i \ne j} A_{i,j}$ for the number of comparisons made by quicksort).
  
    \paragraph{Solution}
  the number of comparisons made by \textsc{QuickSelect} is
  $$
    \sum_{i\not=j}B_{i,j,k}
  $$
  In \textsc{QuickSort}, $i$ will compare with $j$ if $i$ is the ancester of
  $j$ in quicksort tree. But it is not true in \textsc{QuickSelect}, because if $k$ and $j$ 
  not in the same subtree rooted by $i$. The \textsc{QuickSelect} will not enter this subtree because he
  know $k$ in other subtree rooted by $i$'s parent.It means $B_{i,j,k}=0$.
  
  And if $k$ and $j$ in the same subtree rooted by $i$, then \textsc{QuickSelect} will must choose
  the subtree. It means $B_{i,j,k}=1$.
  
  So the answer is the sum of $B_{i,j,k}$
\end{exercise}


\begin{exercise}
   Suppose we use \textsc{QuickSelect} to find the minimum of the array. On expectation,
   how many comparisons will it make? Give an answer that is exact up to additive terms 
   of order $o(n)$.
     You can use the fact that $H_k := 1 + \frac{1}{2} + \frac{1}{3} + \cdots  + \frac{1}{n} = \ln(n) + o(1)$.
     
    \paragraph{Solution}  
  Accroding to Ex7,
  \begin{align*}
    &\sum_{i \not= j} \frac 1 {\max(i,j,1) - \min(i,j,1) + 1} \\
    =& \sum_{i \not= j}\frac 1 {\max(i,j)} \\
    =& 2\sum_{i=1}^n \frac 1 i (i - 1) \\
    =& 2n - H_n
  \end{align*}
  So it makes $2n - \ln n + o(1)$ comparisons.
\end{exercise}

\begin{exercise}
  Derive a formula for $\E_{\pi} [C(\pi,k)]$, up to additive terms of order $o(n)$.
  You might want to introduce $\kappa := k/n$.
  
  
\end{exercise}


\end{document}

\subsection{Other Recurrences}

Suppose we are given the recurrence
\begin{align*}
T(n) := \begin{cases}
0 &  \text{if } n \leq 1 \ , \\
 2T(n/2) + n & \text{else.}
\end{cases} \ .
\end{align*}
By looking at the recursion tree, where were able to show that $T(n) \leq n \log(n)$, 
and in fact $T(n) = n \log(n)$ if $n$ is a power of $2$.

\begin{exercise}
   Extend your proof from the previous exercise. Suppose we are given numbers
   $p_1,\dots, p_k \in (0,1)$ such that $p_1 + \cdots + p_k = 1$. Let
\begin{align*}
T(n) := \begin{cases}
0 &  \text{if } n=1 \ , \\
 n +  \sum_{i=1}^k T(p_i n)  & \text{else.}
\end{cases} \ .
\end{align*}  
It is easy to see that $T(n) \in \Theta( n \log n)$, too. However, I want you to determine
the constant factor. That is, determine a number $\gamma$ such that $T(n) = \gamma n \log n + 
o(n \log n)$. \textbf{Remark.} For this, looking at the recursion tree level by level might not be enough. 
You may have to prove it by induction and see for which value of $\gamma$ it ``goes through''.
\end{exercise}



\end{document}