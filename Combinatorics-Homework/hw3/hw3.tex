\documentclass[UTF8]{ctexart}
\usepackage{amsmath}
\usepackage{amssymb}
\usepackage{amsthm}
\usepackage{graphicx}
\usepackage{CJK}
\usepackage{float}
\usepackage{mdframed}
\providecommand{\abs}[1]{\lvert#1\rvert}
\providecommand{\norm}[1]{\lVert#1\rVert}
\providecommand{\ud}[1]{\underline{#1}}

\newmdtheoremenv{thm}{Theorem}
\newmdtheoremenv{lemma}[thm]{Lemma}
\newmdtheoremenv{fact}[thm]{Fact}
\newmdtheoremenv{cor}[thm]{Corollary}
\newtheorem{eg}{Example}
\newtheorem{ex}{Exercise}
\newmdtheoremenv{defi}{Definition}
\newenvironment{sol}
  {\par\vspace{3mm}\noindent{\it Solution}.}
  {\qed \\ \medskip}

\newcommand{\ov}{\overline}
\newcommand{\ca}{{\cal A}}
\newcommand{\cb}{{\cal B}}
\newcommand{\cc}{{\cal C}}
\newcommand{\cd}{{\cal D}}
\newcommand{\ce}{{\cal E}}
\newcommand{\cf}{{\cal F}}
\newcommand{\ch}{{\cal H}}
\newcommand{\cl}{{\cal L}}
\newcommand{\cm}{{\cal M}}
\newcommand{\cp}{{\cal P}}
\newcommand{\cs}{{\cal S}}
\newcommand{\cz}{{\cal Z}}
\newcommand{\eps}{\varepsilon}
\newcommand{\ra}{\rightarrow}
\newcommand{\la}{\leftarrow}
\newcommand{\Ra}{\Rightarrow}
\newcommand{\dist}{\mbox{\rm dist}}
\newcommand{\bn}{{\mathbb N}}
\newcommand{\bz}{{\mathbb Z}}

\newcommand{\expe}{{\mathsf E}}
\newcommand{\pr}{{\mathsf{Pr}}}
\newenvironment{pf}
  {\par\vspace{3mm}\noindent{\it pf}.}

\setlength{\parindent}{0pt}
%\setlength{\parskip}{2ex}
\newenvironment{proofof}[1]{\bigskip\noindent{\itshape #1. }}{\hfill$\Box$\medskip}

\theoremstyle{definition}
\newtheorem{problem}{Problem}
\newtheorem*{problem*}{Problem}

\pagenumbering{gobble}

\begin{document}

\author{于峥 518030910437}
\title{CS477 Combinatorics: Homework 3}
\date{\today}

\maketitle

\begin{problem}
对任意的$n$,定义$D_k = \binom{n}{k+1} - \binom{n}{k}$。
求$k^*$使得$D_{k^*}$在所有的$D_k$中取值最大。
\begin{sol}
    \begin{align*}
        r(k)=\frac {D_{k}} {D_{k-1}} 
        &= \frac {\binom{n}{k+1} - \binom{n}{k}} {\binom{n}{k} - \binom{n}{k-1}} \\
        &= \frac {\frac {n-2k-1} {k+1} \binom{n}{k}} {\frac {n-2k+1} {k} \binom{n}{k-1}} \\
        &= \frac {(n-2k-1)(n-k+1)} {(n-2k+1)(k+1)}
    \end{align*}
    为了$D_k > 0$, 我们只关心$0 < k \leq \frac n 2$的情况。
    
    在这个范围内,$\frac {n-2k-1} {n-2k+1}$和
    $\frac {n-k+1} {k+1}$都是递减的
    所以在$k$在此范围内比值是递减的。

    因此比值为1的时候,$D_k$最大,$k = \frac 12 (n - \sqrt{n+2})$,因此

    特别的当$n = 1$时, $k^*=0$。
    
    当$n + 2$是完全平方数的时候,$k^* = \frac {n - \sqrt{n+2}} 2, \frac {n - \sqrt{n+2}} 2-1$

    否则
    $$k^*=\left\lfloor \frac {n - \sqrt{n+2}} 2 \right\rfloor$$
    时取到最大值。
\end{sol}
\end{problem}

\newpage
\begin{problem}
集合$A_0=\{0\}$,$A_1=\{1\}$,$A_{n+2} = (A_1 + A_{n+1}) \Delta A_{n}$。证明有无穷多个$n$使得$|A_n|=1$。
    \begin{sol}
        我们可以证明当$n$为$2^k-1$的形式的时候,$|A_n|=1$。

        从另一个角度进行观察,我们将集合看成多项式,将集合操作转化成多项式操作,
        让$B_0(x) = 1, B_1(x) = x$, $B_{n+2}(x)=xB_{n+1}(x) + B_n(x)$。
        其中$B_i(x)$是定义在$GF(2)$上的多项式。

        可以归纳证明$B_i(x)=\sum_{u \in A_i} x^u$,
        首先$n=0,1$时容易验证满足,
        并且若小于等于$n+1$时满足, $A_1 + A_{n+1}$相当于将$A_{n+1}$中所有元素加一,
        而$xB_n(x)$就是将每一项的幂次加一。而集合的对称差操作,多项式相加后由于是定义在
        $GF(2)$上,共有的项会被消去。因此$n+2$时满足。

        进而由于$A_n$必然包含$n$, 所以当$|A_n|=1$时,$B_n(x) = x^n$,
        进而证明$B_i(x)$有如下性质, 令$C_i(x)=B_{i-1}(x)$
        \begin{align*}                
            C_{2n}(x)&=xC_n(x)^2 \\
            C_{2n+1}(x)&=(C_n(x)+C_{n+1}(x))^2
        \end{align*}
        
        同样进行归纳证明法,$n=1,2,3,4$可以验证是满足的,若$1\sim 2n$都满足,那么
        \begin{align*}
            C_{2n+1}(x)
            &=xC_{2n}(x)+C_{2n-1}(x) \\
            &=x^2C_{n}(x)+(C_{n-1}(x)+C_n(x))^2 \\
            &=C_n(x)^2 + (x^2C_n(x)+C_{n-1}(x))^2\\
            &=C_n(x)^2 + C_{n+1}(x)^2 \\
            &=(C_n(x) + C_{n+1}(x))^2 \\
        \end{align*}
        \begin{align*}
            C_{2n+2}(x)
            &=xC_{2n+1}(x)+C_{2n}(x) \\
            &=x(C_n(x)+C_{n+1}(x))^2+xC_n(x)^2\\
            &=xC_{n+1}(x)^2
        \end{align*}
        因此$2n+1,2n+2$也满足这一性质。不难发现$C_2n(x)$的项数永远和$C_n(x)$相同,而
        $C_1(x)$只有一项,所以$C_{2^k}(x)$项数为一,进而$B_{2^k-1}(x)$项数为1,所以
        当$n$为$2^k-1$的形式的时候,$|A_n|=1$。

    \end{sol}
\end{problem}

\begin{problem}
给定正整数$n, r, s$,有多少个$[2n]$的子集恰好包含$r$个奇数和$s$个偶数,且没有任何两个相邻的元素?
    \begin{sol}
        \begin{lemma}
            将$r$个元素放到$n$个位置中,每个位置可以放任意个,方案数为:
            $$\binom{n+r-1}{n-1}$$
        
            \begin{pf}
                假设将元素放置好后,在每个位置再额外放一个元素。这就相当于将
                $n+r$个元素放到$n$个位置中,每个位置至少放一个的方案。我们可以考虑将这些
                元素排成一列,在其中插入$n-1$个板,分成$n$份的方案为$\binom{n+r-1}{n-1}$。
            \end{pf}    
        \end{lemma}
        我们可以这样构造集合,首先可以将问题看成将$r$个$10$,$s$个$01$,$n-r-s$个$00$
        拼成长度为$2n$的$01$串。我们首先只考虑$10$和$00$,有$\binom{n-s}{r}$种方法,将他们
        拼起来。然后将$01$插入到其中,由于每个$10$的前面不能放$01$,这相当于有
        $s$个元素,放到$n-s-r+1$个位置中,每个位置可以放任意个,根据\textbf{Lemma 1}, 这有$\binom{n-r}{n-s-r}$种方法。
        得到答案:
        $$\binom{n-r}{s}\binom{n-s}{r}$$
    \end{sol}
\end{problem}

\begin{problem}
证明:平面上存在$13^3 = 2197$个不同的点,使得它们之间有至少12345对的距离为13。
\begin{sol}
    这个问题可以转化成考虑找到一个平面上$n$个点,$m$条边的图,使得每条边的长度都相同。本题要求$n\leq 2197$,
    $m \geq 12345$。

    我们可以考虑Hypercube graph是可能满足要求的,因为Hypercube graph有$2^n$个点,$2^{n-1}n$条边,边的密度比较高。
    并且在平面上画出 Hypercube graph,并且让每条边的长度相同是容易(我们让边的长度是单位长度)。因为首先$Q_1$容易画出,而要画出$Q_n$
    只需将$Q_{n-1}$平移单位距离再将对应的点连上即可。
    
    然而Hypercube依然难以满足要求,我们可以对Hypercube进行一些修改。令$G_1$为一个边长为单位长度的
    等边三角形,$G_{n+1}$由$G_{n}$通过以下步骤构造出来:

    a) 将$G_n$向两个方向移动单位距离,并且两个方向的夹角为$60$度。

    b) 把生成的三个图的对应点连起来,可以发现对应点构成一个等边三角形。

    还需要说明的细节是,一定存在这样的方向让平移后的点与之前的不重合。这是因为方向的选择有无限种,
    然而平面上的点是有限的。

    因此$G_n$的所有边长度相同,而$G_n$的点数$V_n=3^n$, 边数满足$E_n = 3E_{n-1} + V_n=n3^n$。
    注意到边和点的比值为$n$, 比Hypercube更强。 算出当$n=7$时,有$2187$个点,$15309$条边,满足题目要求。

    这种构造的本质就是将等边三角形的图作笛卡尔积,事实上我们也可以通过对不同的图形作笛卡尔积
    来达到同样的目的。

    关于"The maximum number of occurrences of the same distance among n points in the plane"
    似乎是个未解问题,我查到的上界是$cn^{4/3}$。用这题笛卡尔积的思路能做到的最好情况
    应该都是$kn$的。其中$k$是初始图形的点和边的比例。
    
\end{sol}
\end{problem}

\begin{problem}
如果人和人之间的朋友关系是相互的。证明:在任意100个人中,有两个人使得他们的共同的朋友有偶数个。
\begin{sol}
    我们证明任意偶数个人中存在两个人的公共朋友数为偶数。

    1) 若存在一个人$P$有奇数个朋友$Q_1, Q_2, \dots, Q_n$, 设$PQ_i$
    为$P$与$Q_i$的公共朋友数。如$Q_i$之间没有朋友关系,那么所有$PQ_i$为$0$是偶数。

    
    $Q_i$之间每增加一对朋友关系,那么$PQ_i(i=1,2,\dots,n)$就有两个数的奇偶性改变。
    然而想要让$PQ_i$全部为奇数必须只能进行奇数次奇偶性改变。因此最后必然存在一个$Q_i$
    与$P$的公共朋友为偶数。

    2) 若所有人的朋友都是偶数个,假设$P$的朋友集合为$A$, 其他人的集合为$B$,
    则$|A|$是偶数,$|B|$是奇数。
    
    $\forall Q \in B$, 如果$Q$在$A$中有偶数个朋友,
    那么$PQ$有偶数个公共朋友。
    
    否则$Q$在$A$中有奇数个朋友,那么$Q$在$B$中也有奇数个朋友,
    由于$Q$是任选的,那么把$B$看成一个奇数个点的子图,每个点的度数都是奇数,那么度数
    和也是奇数,然而无向图的度数和必然为偶数,产生矛盾。
\end{sol}
\end{problem}

\begin{problem} (*)
在一个$n \times n$的方格表的每个格子里填上一个正整数,使得每个数可以整除它上面的数(如果有的话),也可以整除它右边的数(如果有的话),并且整个表右上角格子
里填的数是$T$。当$T=2$、$T=4$、$T=30$、$T = 1000$时,这样的填法各有多少种?

\begin{sol}
    我们可以这样考虑,
    设$f_n(T)$为该问题答案。将$T$进行质因数分解,
    例如$T=p_1^{k_1}p_2^{k_2}\dots p_n^{k_n}$。 
    可以发现不同质因子之间没有影响,即$f_n(T)=f_n(p_1^{k_1})f_n(p_2^{k_2})\dots f_n(p_n^{k_n})$。

    进而我们只需要考虑$p^k$形式的$T$, 由于当$k$相同时得到的答案相同,所以我们将
    问题转化成在方格中填$0$到$k$让每行每列都递减,为了去掉右上角
    必须填$k$的限制,记$h_n(k)=f_n(p^k)$, 可以考虑计算$g_n(k)=\sum_{i=0}^kh_n(i)$,即计算
    前缀和。

    对于这$k$条路径,我们考虑将第$i$条向上和向右都平移$i-1$格,那么这$k$条路径将严格不相交。题目
    转化成了从$(0,0),(-1,1)\dots(1-k,k-1)$走到$(n,n),(n-1,n+1),(n-2,n+2)\dots(n-k+1,n+k-1)$有
    多少种走法让路径不相交。使用LGV引理。

    \begin{lemma}
        对于有向无环图,
        $e(u, v)$表示从$u$走到$v$所有路径的边权和。

        $\omega(P)$表示$P$这条路径上所有边的边权之积。

        起点集合$A$,是有向无环图点集的一个子集,大小为$n$。

        终点集合$B$,也是有向无环图点集的一个子集,大小也为$n$。

        一组$A \rightarrow B$的不相交路径$S$: $S_i$是一条从$A_i$到$B_{\sigma(S)_i}$的路径
        ($\sigma(S)$是一个排列),对于任何$i \not= j$,$S_i$和$S_j$没有公共顶点。

        $N(\sigma)$表示排列$\sigma$的逆序对个数。

        $$
        M = 
        \begin{pmatrix}
            e(A_1,B_1) & e(A_1,B_2) & \dots & e(A_n,B_n)\\
            e(A_2,B_1) & e(A_2,B_2) & \dots & e(A_2,B_n)\\
            \vdots & \vdots & \ddots & \vdots \\
            e(A_n,B_1) & e(A_n,B_2) & \dots & e(A_n,B_n)
        \end{pmatrix}
        $$
        $$\det(M)=\sum_{S:A\rightarrow B}(-1)^{N(\sigma(S))}\prod_{i=1}^n \omega(S_i)$$
    \end{lemma}
    将边权设为$1$,由于对应关系是$i \rightarrow i$,所以答案就是$g_n(k)=\det(M)$,

    当$k=0$时, 
    $$g_n(0) = h_n(0) = 1$$。

    当$k=1$时, 
    \begin{align*}
        g_n(1) &= \binom{2n}{n} \\
        h_n(1) &= \binom{2n}{n} - 1
    \end{align*}
    
    因为对于任意一条从左下到右上的
    路径的上方放$1$,下方放$0$,这对应了一种放置方法。而每种放置方法的边界也可以看成
    一条路径。

    当$k=2$时, 
    $$
    M = 
    \begin{bmatrix}
        \binom{2n}{n} & \binom{2n}{n-1} \\
        \binom{2n}{n-1} & \binom{2n}{n} 
    \end{bmatrix}
    $$
    \begin{align*}
    g_n(2) &= \binom{2n}{n}^2 - \binom{2n}{n-1}^2 \\
    h_n(2) &= \frac {1} {n+1}\binom{2n}{n}\binom{2n}{n-1}
    \end{align*}

    当$k=3$时, 
    $$
    M = 
    \begin{bmatrix}
        \binom{2n}{n} & \binom{2n}{n-1} & \binom{2n}{n-2} \\
        \binom{2n}{n-1} & \binom{2n}{n} & \binom{2n}{n-1} \\
        \binom{2n}{n-2} & \binom{2n}{n-1} & \binom{2n}{n} 
    \end{bmatrix}
    $$
    \begin{align*}
        g_n(3) &= \frac 2 {(n+1)^2(n+2)}\binom{2n}{n}\binom{2n+1}{n}\binom{2n+2}{n} \\
        h_n(3) &= \frac 1 {n+1}\binom{2n}{n}\binom{2n+1}{n} \left[ \frac {2} {(n+1)(n+2)}\binom{2n+2}{n}-1 \right]
    \end{align*}

    对于任意的$k$,有
    $$g_n(k)=\prod_{i=1}^k\prod_{j=1}^n\frac {n+i+j-1} {i+j-1}$$

    因此$T=2$时,答案为
    $$
        h_n(1) = \binom{2n}{n} - 1
    $$
    $T=4$时,答案为
    $$
        h_n(2) = \frac {1} {n+1}\binom{2n}{n}\binom{2n}{n-1}
    $$
    $T=30=2\times 3\times 5$时,答案为
    $$
        h_n(1)^3 = \left( \frac {1} {n+1}\binom{2n}{n}\binom{2n}{n-1} \right)^3
    $$
    $T=1000=2^3 \times 5^3$时,答案为
    $$
        h_n(3)^2 = \left( \frac 1 {n+1}\binom{2n}{n}\binom{2n+1}{n} \left[ \frac {2} {(n+1)(n+2)}\binom{2n+2}{n}-1 \right] \right)^2
    $$
\end{sol}
\end{problem}

\end{document}

