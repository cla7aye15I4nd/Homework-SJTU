\documentclass[UTF8]{ctexart}
\usepackage{amsmath}
\usepackage{amssymb}
\usepackage{amsthm}
\usepackage{graphicx}
\usepackage{CJK}
\usepackage{float}
\usepackage{mdframed}
\providecommand{\abs}[1]{\lvert#1\rvert}
\providecommand{\norm}[1]{\lVert#1\rVert}
\providecommand{\ud}[1]{\underline{#1}}

\newmdtheoremenv{thm}{Theorem}
\newmdtheoremenv{lemma}[thm]{Lemma}
\newmdtheoremenv{fact}[thm]{Fact}
\newmdtheoremenv{cor}[thm]{Corollary}
\newtheorem{eg}{Example}
\newtheorem{ex}{Exercise}
\newmdtheoremenv{defi}{Definition}
\newenvironment{sol}
  {\par\vspace{3mm}\noindent{\it Solution}.}
  {\qed \\ \medskip}

\newcommand{\expe}{{\mathsf E}}
\newcommand{\pr}{{\mathsf{Pr}}}


\setlength{\parindent}{0pt}
%\setlength{\parskip}{2ex}
\newenvironment{proofof}[1]{\bigskip\noindent{\itshape #1. }}{\hfill$\Box$\medskip}

\theoremstyle{definition}
\newtheorem{problem}{Problem}
\newtheorem*{problem*}{Problem}

\pagenumbering{gobble}

\begin{document}

\title{CS477 Combinatorics: Homework 11}
\date{May. 13, 2020}

\maketitle

\begin{problem}
证明:对任意的$k$,都存在一个$N$,使得对平面上任何$N$个没有三点共线的点,
都能找到$k$个构成一个凸$k$边形。
\end{problem}

\begin{problem}
证明:
\[
r(k, 4) \in \Omega(k^{1.49258367}).
\]
\begin{sol}
  考虑对$K_n$随机染色,以$p$的概率染上黄色,$q=1-p$的概率染上蓝色,那么假设
  黄色$K_k$的数量为$X$, 存在蓝色$K_4$的数量为$Y$。
  那么
  $$
  E(X) = \binom{n}{k}p^{\binom{k}{2}} ~~~
  E(Y) = \binom{n}{4}q^{\binom{4}{2}} 
  $$
  我们让$q = 4n^{-\frac 2 3}$, 那么
  \begin{align*}
    E(Y) \leq \frac {n^4} {24} q^{6} \leq \frac 1 2
  \end{align*}
  同时
  \begin{align*}
    E(X) 
    &\leq \frac {n^k} {k!} (1-4n^{-\frac 2 3})^{\frac{k(k-1)} {2}} \\
    &\leq \frac {n^k} {k!} e^{-n^{\frac 2 3}k^2}
  \end{align*}
\end{sol}
\end{problem}

\begin{problem}
证明:对任意的$k,l$,都存在图$G$使得$\chi(G) > k, g(G) > l$。
\begin{sol}
  由于一个图的色数为$k$意味着将它分为了$k$个不相交的独立集,因此不难得到
  $$
  \alpha (G) < \frac {|G|} k \Rightarrow \chi(G) > k
  $$
  假设图$G$上每条边以$p$的概率出现,我们记事件$\{ \alpha (G) \geq \frac {|G|} k \}$
  为$X$, 事件$\{ g(X) \leq l \}$为$Y$。那么我们只需使$P(X) + P(Y) < 1$即可。

  首先考虑$P(X)$, 图$G$上每条边以$p$的概率出现,记$r=\frac {n} {k}$, 那么
  $$
  P(X) \leq \binom{n}{r}(1-p)^{\binom{r}{2}} \leq 2^ne^{-p\binom {r} {2}}
  $$
  我们只要让$p \geq \frac n {\binom{r}{2}} = \frac {2k^2} {n - k}$即可。容易看到只要$n$足够大,
  我们就能让$P(X)$小于任何正常数。

  再考虑$P(Y)$, 当$n > l$时, $P(Y)$不可能为$1$, 此时无论$P(Y)$为何值,
  我们都能取足够大的$n$,使得$P(X) + P(Y) < 1$, 因此命题成立。
\end{sol}
\end{problem}

\begin{problem}
证明:对任意图$G=(V, E)$,都存在$V$的两个不相交的子集$A$和$B$,使得$A$和$B$之间的边至少有$|E|/2$条。
\begin{sol}
  \paragraph{算法}按某种顺序对点黑白染色,每次染一个点时如果该点和黑点连边
  比和白点连边多,那么染成白色,否则染成黑色。
  \paragraph{概率方法} 随机对点二染色后,异色点之间的边期望有$|E|/2$,因
  此存在这样的划分。
\end{sol}
\end{problem}

\begin{problem}
如果没有孤立点,将上面的$|E|/2$改成$|E|/2 + |V|/6$。
\begin{sol}
  
\end{sol}
\end{problem}

\begin{problem}
当$G$是连通图时,将上面的$|E|/2$改成$|E|/2 + |V|/4 - 1$。
\end{problem}

\end{document}
