\documentclass[UTF8]{ctexart}
\usepackage{amsmath}
\usepackage{amssymb}
\usepackage{amsthm}
\usepackage{graphicx}
\usepackage{CJK}
\usepackage{float}
\usepackage{mdframed}
\providecommand{\abs}[1]{\lvert#1\rvert}
\providecommand{\norm}[1]{\lVert#1\rVert}
\providecommand{\ud}[1]{\underline{#1}}

\newmdtheoremenv{thm}{Theorem}
\newmdtheoremenv{lemma}[thm]{Lemma}
\newmdtheoremenv{fact}[thm]{Fact}
\newmdtheoremenv{cor}[thm]{Corollary}
\newtheorem{eg}{Example}
\newtheorem{ex}{Exercise}
\newmdtheoremenv{defi}{Definition}
\newenvironment{sol}
  {\par\vspace{3mm}\noindent{\it Solution}.}
  {\qed \\ \medskip}

\newcommand{\ov}{\overline}
\newcommand{\ca}{{\cal A}}
\newcommand{\cb}{{\cal B}}
\newcommand{\cc}{{\cal C}}
\newcommand{\cd}{{\cal D}}
\newcommand{\ce}{{\cal E}}
\newcommand{\cf}{{\cal F}}
\newcommand{\ch}{{\cal H}}
\newcommand{\cl}{{\cal L}}
\newcommand{\cm}{{\cal M}}
\newcommand{\cp}{{\cal P}}
\newcommand{\cs}{{\cal S}}
\newcommand{\cz}{{\cal Z}}
\newcommand{\eps}{\varepsilon}
\newcommand{\ra}{\rightarrow}
\newcommand{\la}{\leftarrow}
\newcommand{\Ra}{\Rightarrow}
\newcommand{\dist}{\mbox{\rm dist}}
\newcommand{\bn}{{\mathbb N}}
\newcommand{\bz}{{\mathbb Z}}

\newcommand{\expe}{{\mathsf E}}
\newcommand{\pr}{{\mathsf{Pr}}}


\setlength{\parindent}{0pt}
%\setlength{\parskip}{2ex}
\newenvironment{proofof}[1]{\bigskip\noindent{\itshape #1. }}{\hfill$\Box$\medskip}

\theoremstyle{definition}
\newtheorem{problem}{Problem}
\newtheorem*{problem*}{Problem}

\pagenumbering{gobble}

\begin{document}

\title{CS477 Combinatorics: Homework 8}
\author{于峥 518030910437}
\date{\today}

\maketitle

\begin{problem}
$\pi$和$\sigma$是两个$[2020]$上的排列,各自的圈结构中恰好有2000个圈。
确定$\pi \circ \sigma$的圈数的所有可能值。
\begin{sol}
  在求所有的可能值之前,我们不妨先确定圈数的上下界。由于我们知道一个对换可以分裂或合并一个环。
  
  如果我们想让圈数最小,我们要尽可能去合并环。由于环结构中恰好有2000个圈,我们将$\pi$分解成对换,我们取分解后得到对换数量最少的,那么
  每个圈可以贡献圈数大小减一个对换。所以我们有20个对换。现在我们希望每个对换作用在$\sigma$上后都能完成一次合并。这显然是可以做到的,
  例如我们让$\sigma$包含一个大小为$21$的环$(1,2\cdots 21)$,和1999个大小为1的环。那么我们只要简单的让$\pi$为$(22,23)(24,25)\cdots(60, 61)$
  即可。此时$\pi\circ\sigma$有$1980$个环。
  
  同样要让环最大只需要让$\pi$为$(1,2)(2,3)\cdots(20,21)$,也就是$\sigma^{-1}$。此时有$2020$个环。由于我们注意到$\pi$中
  20个对换每一个都能让圈的数量加一或减一。因此$1980$到$2020$之间的偶数都是可能做到的。
  
\end{sol}
\end{problem}

\begin{problem}
$[n]$的所有置换中有多少个恰好有1个圈?恰好有2个、3个、$n-2$个、$n-1$个、$n$个?

\begin{sol}
  令$f_{n,m}$表示$[n]$的所有置换中恰有$m$个圈的数量,我们可以通过枚举$1$所在
  圈的大小进行递推求解
  $$
  f_{n,m}=\sum_{i=1}^{n-m+1}(i-1)!\binom{n-1}{i-1}f_{n-i,m-1}
  $$
  当$m=1$时求解的是大小为$n$的轮换数量,所以$f_{n,1}=(n-1)!$。
  
  当$m=2$时,
  \begin{align*}
    f_{n,2}
    &=\sum_{i=1}^{n-1}(i-1)!\binom{n-1}{i-1}f_{n-i,1}\\
    &=(n-1)!\sum_{i=1}^{n-1}\frac 1 {n-i}\\
    &=H_{n-1}(n-1)!
  \end{align*}

  当$m=3$时,
  \begin{align*}
      f_{n,3}
      &=\sum_{i=1}^{n-2}(i-1)!\binom{n-1}{i-1}f_{n-i,2}\\
      &=(n-1)!\sum_{i=1}^{n-2}\frac {H_{n-i-1}} {n-i}\\
      &=(n-1)!\sum_{i=1}^{n-1}\frac {H_{n-i-1}} {n-i}\\
      &=(n-1)!\sum_{i=1}^{n-1}\frac {H_{i-1}} {i}\\
      &=\frac 1 2(n-1)![H_{n-1}^2-H^{(2)}_{n-1}]
  \end{align*}

  当$m=n$时,全都是大小为$1$的圈$f_{n,n}=1$。

  当$m=n-1$时,有一个大小为$2$的圈,其他都为$1$,所以$f_{n,n-1}=\binom{n}{2}$。

  当$m=n-2$时,可以有一个大小为$3$的圈,其他都为$1$或者由两个大小为$2$的圈,所以
  \begin{align*}
    f_{n,n-2}
    &=2\binom{n}{3}+\frac 1 2 \binom{n}{2}\binom{n-2}{2} \\
    &=2\binom{n}{3}+3\binom{n}{4} \\
  \end{align*}
\end{sol}
\end{problem}

\begin{problem}
等概率地随机地在$[n]$的所有置换中取一个。

(a) 1和2在同一个圈中的概率是多少?

(b) 1所在的圈长度为$k$的概率是多少?

(c) 1所在的圈的长度的期望是多少?

(d) 圈的个数的期望是多少?

(e) 每个圈至少包含$[k]$中一个元素的概率是多少?
\begin{sol}

  (a) 计算1和2在同一个圈中的置换数量, 枚举1所在圈的大小
  \begin{align*}
      P
      =&\frac 1 {n!} \sum_{i=2}^n(i-1)!(n-i)!\binom{n-2}{i-2}\\
      =&\frac 1 {n(n-1)} \sum_{i=2}^n i-1 \\
      =& \frac 1 2
  \end{align*}

  (b) 1所在圈大小为$k$有$(k-1)!\binom{n-1}{k-1}(n-k)!=(n-1)!$种,概率为$\frac 1 n$。

  (c) 由上一问,期望$E=\frac 1 n \sum_{i=1}^n\frac 1 i= \frac {n+1} 2$。

  (d) 沿用第二问的记号, 不难得到
  $$
    E_n=\sum_{m=1}^n\frac {mf_{n,m}} {n!} 
  $$
  发现$\sum_{m=1}^n mf_{n,m}$代表的意义就是所有置换的每个圈都统计一次贡献,但是我们也可以认为是每个圈中最小的那个元素作了一次贡献。那么
  $1$在所有置换中都做了一次贡献,即贡献为$n!$。我们记$g_n=\sum_{m=1}^n mf_{n,m}$, 那么除了1以外其他元素所作的贡献我们可以先
  枚举1所在环大小,那么剩下的元素贡献就成为了一个子问题。因此
  $$
  g_n = n! + \sum_{m=1}^n (m-1)!\binom{n-1}{m-1}g_{n-m} = n! + (n-1)!\sum_{m=1}^{n-1}\frac {g_m} {m!}
  $$
  对比$E_n$和$g_n$, 可以发现
  $$
  E_n = 1 + \frac 1 n \sum_{m=1}^{n-1}E_m
  $$
  并且$E_1=1$, 对比调和级数恒等式
  $$\sum _{{k=1}}^{n}H_{k}=(n+1)H_{n}-n$$
  可以发现两者相同,我们得到$E_n=H_n$。

  (e) 令$h_{n,k}$代表大小为$n$的置换每个圈包含$[k]$中一个元素的个数。我们首先有$h_{n,0}=0, h_{0,0}=1$。
  当$n>k>0$时,我们枚举$[k]$中最小元素所在圈的大小和在$[k]$中的元素个数。
  \begin{align*}
    h_{n,k}&=\sum_{i=1}^{k}\sum_{j=0}^{n-k}\binom{k-1}{i-1}\binom{n-k}{j}(i+j-1)!h_{n-i-j,k-i}\\
  \end{align*}

  上面这个式子不会算,我们可以组合解释,认为是先让$[k]$中的元素组成一个置换,然后放入其他元素,放入
  第一个元素有$k$种方法,然后是$k+1,k+2,\cdots,n-1$。
  
  所以$h_{n,k}=k!\cdot k \cdot (k+1)\cdots(n-1)=k(n-1)!$,因此概率为$k/n$。
  
  \paragraph{Tips}:上面的题目可以用这样的性质做,一个置换可以表示成若干个轮换,如果我们把每个轮换
  的最小元素放在最后,然后按最小元素排列。例如$(11,7,5,9,1)(6,3,2)(4,8,10,3)$我们将括号去掉之后依然
  能够得到原排列是什么样的。
\end{sol}
\end{problem}

\begin{problem} *
将$\mathbf{Z}_n$(模$n$的同余系$0, 1, \dots, n-1$组成的环)的元素排在一个等分成$n$格的圆周上,每个格子一个元素,使得如果$a+1$顺时针在$b$之前一格,那么$b+1$顺时针在$a$之前一格。对什么样的
$n$,必定存在$x$使得$x+1$在$x$之前一格?

\paragraph{Claim} 当$n$不是4的倍数时,必定存在$x+1$在$x$前一格。

(a) 当$n=4k$,我们这样进行构造。如果轮换为$a=(a_0, a_1, a_2, \cdots, a_{n-1})$,让
$a_0=n-1, a_1 = n-4, a_2 = n-3, a_3=n-2$, 其他$a_i = a_{i-4} - 4$。

我们先按下标四个为一组,对组内的关系进行检查,每一组都形如
$4k-1,4k-4,4k-3,4k-2$。$4k-1$在$4k-4$之前推出$4k-3$在$4k-2$之前,符合要求。
$4k-4$在$4k-3$之前推出$4k-1$在$4k-5$之前。不难发现$4k-1$后继的确是$4k-5$,
因为我们有$a_i = a_{i-4} - 4$。这样我们顺便也完成了相邻组的检查,因此该构造满足要求。

(b) 当$n$为奇数时,由于我们当我们想让$a+1$在$b$之前,必然有$b+1$在
$a$之前, 这意味着我们每次都会为两个不同的元素选择后继,而$n$为奇数,那么
最后只剩一个元素没有决定后继时一定会出现矛盾,所以必定存在$x$使得$x+1$在$x$之前一格。

(c) 当$n=4k+2$时,假设满足条件的轮换为$\pi$。令$\sigma$为轮换$(0,1,2,\cdots,n-1)$。
那么$\pi\sigma$将会是$\frac n 2$个不相交的对换。因为如果$(\pi\sigma)(a)=\pi(a+1)=b$,
那么必有$(\pi\sigma)(b)=\pi(b+1)=a$。进而注意到$\pi = (\pi\sigma)\sigma^{-1}$。
由于$\sigma^{-1},\pi$都是一个环,而$\pi\sigma$将会是奇数个的对换,因此等式右边是偶置换,左边是奇置换。
因此不可能存在满足条件的$\pi$.

\paragraph{PS} 感觉这题也不是很复杂QAQ,最后一种情况想了好久。



\end{problem}

\end{document}


