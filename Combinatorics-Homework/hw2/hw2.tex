\documentclass[UTF8]{ctexart}
\usepackage{amsmath}
\usepackage{amssymb}
\usepackage{amsthm}
\usepackage{graphicx}
\usepackage{CJK}
\usepackage{float}
\usepackage{mdframed}
\providecommand{\abs}[1]{\lvert#1\rvert}
\providecommand{\norm}[1]{\lVert#1\rVert}
\providecommand{\ud}[1]{\underline{#1}}

\newmdtheoremenv{thm}{Theorem}
\newmdtheoremenv{lemma}[thm]{Lemma}
\newmdtheoremenv{fact}[thm]{Fact}
\newmdtheoremenv{cor}[thm]{Corollary}
\newtheorem{eg}{Example}
\newtheorem{ex}{Exercise}
\newmdtheoremenv{defi}{Definition}
\newenvironment{sol}
  {\par\vspace{3mm}\noindent{\it Solution}.}
  {\qed \\ \medskip}

\newenvironment{pf}
  {\par\vspace{3mm}\noindent{\it pf}.}

\newcommand{\ov}{\overline}
\newcommand{\ca}{{\cal A}}
\newcommand{\cb}{{\cal B}}
\newcommand{\cc}{{\cal C}}
\newcommand{\cd}{{\cal D}}
\newcommand{\ce}{{\cal E}}
\newcommand{\cf}{{\cal F}}
\newcommand{\ch}{{\cal H}}
\newcommand{\cl}{{\cal L}}
\newcommand{\cm}{{\cal M}}
\newcommand{\cp}{{\cal P}}
\newcommand{\cs}{{\cal S}}
\newcommand{\cz}{{\cal Z}}
\newcommand{\eps}{\varepsilon}
\newcommand{\ra}{\rightarrow}
\newcommand{\la}{\leftarrow}
\newcommand{\Ra}{\Rightarrow}
\newcommand{\dist}{\mbox{\rm dist}}
\newcommand{\bn}{{\mathbb N}}
\newcommand{\bz}{{\mathbb Z}}

\newcommand{\expe}{{\mathsf E}}
\newcommand{\pr}{{\mathsf{Pr}}}


\setlength{\parindent}{0pt}
%\setlength{\parskip}{2ex}
\newenvironment{proofof}[1]{\bigskip\noindent{\itshape #1. }}{\hfill$\Box$\medskip}

\theoremstyle{definition}
\newtheorem{problem}{Problem}
\newtheorem*{problem*}{Problem}

\pagenumbering{gobble}

\begin{document}

\title{CS477 Combinatorics: Homework 2}
\author{于峥 518030910437}
\date{\today}

\maketitle

\begin{problem}
$n$个硬币扔在地上,每个随机地正面向上或者反面向上,在所有可能的情况中,正反硬币差的绝对值的平均数是多少?
\begin{sol}
  枚举正面硬币的个数然后直接进行计算,
  $$
  \begin{aligned}
    \frac 1 {2^n}\sum_{i=0}^n|n-2i|\binom{n}{i}
    &=\frac 1 {2^{n-1}}\sum_{i=0}^{\lfloor\frac n 2\rfloor}(n-2i)\binom{n}{i} \\
    &=\frac n {2^{n-1}}(1+\sum_{i=1}^{\lfloor\frac n 2\rfloor}\binom{n}{i} - 2\binom{n-1}{i-1})\\
    &=\frac n {2^{n-1}}(1+\sum_{i=1}^{\lfloor\frac n 2\rfloor}\binom{n-1}{i} - \binom{n-1}{i-1})\\
    &=\frac n {2^{n-1}}\binom{n-1}{\lfloor\frac n 2\rfloor}
  \end{aligned}  
  $$
  特别的,$n=0$时平均数为$0$, 否则为$\frac n {2^{n-1}}\binom{n-1}{\lfloor\frac n 2\rfloor}$。
\end{sol}
\end{problem}

\begin{problem} 证明:对任意自然数$n$和$m$,
\[
\sum_{k=0}^n \binom{n}{k}\binom{k}{m} = \binom{n}{m} 2^{n-m}.
\]
\begin{sol}
  \textbf{组合证明}:左边式子可以认为是从$n$个人中选若干个人开会,并且要从这几个人中
  选出$m$个代表。而右边的式子则是先选出$m$个代表,然后在剩下的$n-m$个人中选择要开会的
  非代表人员。

  \textbf{生成函数}:
  用左式构造关于$m$的生成函数
  \begin{align*}
    &\sum_{m=0}^\infty\sum_{k=m}^n\binom{n}{k}\binom{k}{m}x^m  \\
    =& \sum_{k=0}^\infty\sum_{m=0}^k\binom{n}{k}\binom{k}{m}x^m  \\
    =& \sum_{k=0}^\infty \binom{n}{k}\sum_{m=0}^k\binom{k}{m}x^m  \\
    =& \sum_{k=0}^\infty \binom{n}{k}(1+x)^k \\
    =& (2 + x)^n
  \end{align*}

  其中$x^m$的系数正是$ \binom{n}{m} 2^{n-m}$。
\end{sol}
\end{problem}

\begin{problem}
证明:对任意正整数$n$, $a$, $b$,
\[
\sum_k \binom{k}{a} \binom{n-k}{b} = \binom{n+1}{a+b+1}.
\]
\begin{sol}
  \textbf{组合证明}:左边的式子可以看作从一路$n+1$个人的队伍中先选出第$k+1$个人,
  然后从前面$k$个人中选$a$个人,从后面$n-k$个人中选$b$个人。而右边就是直接从这
  $n+1$个人中选$a+b+1$个人。

  右侧的每一个选择,可以找到被选出某个人所在队伍位置$k$前有$a$个人被选出,由于这样的位置
  唯一, 所以这对应了左边的唯一种选择,且不会有右侧的两种选择对应到左侧的同一选择。
  
\end{sol}
\end{problem}

\begin{problem}
证明:对任意正整数$n$,
\[
\sum_k \binom{n}{2k} \binom{2k}{k} 2^{n-2k} = \binom{2n}{n}.
\]
\begin{sol}
  \textbf{组合证明}:
  右侧式子可认为是两个长为$n$的二进制串上的1的个数共有n个的方案数。将第一个二进制串
  第$i$位与另一个二进制串的第$i$位进行配对,假设这些配对中$(0,0)$有$x$对,$(1,1)$有$y$对,
  $(0,1)$和$(1,0)$共$z$对。那么有
  \begin{equation*}
    \left\{
      \begin{aligned}
        x + y + z = n \\
        z + 2x = n
      \end{aligned}
    \right.
  \end{equation*}
  可以得到$x=y$, 所以任意一种选择都有这样的性质,记$x$为$k$,
  这意味着我们可以构造二进制串达到同样的效果:

  1. 首先选出$2k$个位置, 表示这两个串中对应位置的数相同的个数,然后再选$k$个位置放1,
  其他位置放0,这一步放了$2k$个1。

  2. 第一个串剩下$n-2k$个位置随意填$0$和$1$,但是第二个串对应位置放相反的数,
  这一步放了$n-2k$个1。

  而这个过程就是左边式子的选择方式。
\end{sol}
\end{problem}

\begin{problem}
证明:对任何正整数$n$, $m$,
\[
\sum_r \binom{2n}{2r-1} \binom{r-1}{m-1} = \binom{2n-m}{m-1}2^{2n-2m+1}.
\]
\begin{sol}
\begin{align*}
  &\sum_r \binom{2n}{2r-1} \binom{r-1}{m-1} \\
  =& \sum_r \sum_k\binom{2n-k-1}{r-1}\binom{k}{r-1}\binom{r-1}{m-1} \\
  =& \sum_r \sum_k\binom{2n-k-1}{r-1}\binom{k-m+1}{r-m}\binom{k}{m-1} \\
  =& \sum_k \binom{k}{m-1} \sum_r \binom{2n-k-1}{r-1}\binom{k-m+1}{r-m} \\
  =& \sum_k \binom{k}{m-1} \binom{2n-m}{k} \\
  =& \binom{2n-m}{m-1}2^{2n-2m+1}
\end{align*}
\end{sol}
\end{problem}

\begin{problem}
Euler函数$\phi: \mathbb{Z}^+ \to \mathbb{Z}^+$定义是:$\phi(n)$是$[n]$中和$n$互质的数的个数。
定义
\[
f(n) = \sum_{i=1}^n \phi(n).
\]

(a) 找到一个正的常数$k$,使得$f \in \Theta(n^k)$;

(b)(*) 找到一个正常数$C$,使得$f \sim Cn^k$。

\begin{sol}
  (a) 
  \begin{lemma}
    $n=\sum_{d|n}\phi(d)$
    \begin{pf}
      考虑将$1\sim n$按进行分类, 定义等价类
      $$
      \hat m_d = \{ 0 < m \leq n : (m, n) = d\}
      $$
      显然$\hat m_d$两两不交,且如果$d|n$则$|\hat m_d|=\phi(\frac n d)$。
      所以 
      $$n = \sum_{d|n}|\hat m_d| =\sum_{d|n}\phi(\frac n d)=\sum_{d|n}\phi(d)$$
    \end{pf}
  \end{lemma}
  \begin{lemma}
    $[n=1]=\sum_{d|n}\mu(d)$
    \begin{pf}
      $n=1$时等式显然成立。考虑$n$的质因子集合$S$,奇数大小的集合与偶数大小的集合数量相同。
      $$
      \sum_{d|n}\mu(d)
      =\sum_{T\in S }(-1)^{|T|}
      =0
      $$
    \end{pf}
  \end{lemma}

  \newpage
  \begin{lemma}
    $\phi(n)=\sum_{d|n}\mu(d)\frac n d$
    \begin{pf}
      \begin{align*}
        \sum_{d|n}\mu(d)\frac n d 
        &= \sum_{d|n}\mu(d)\sum_{pd|n} \phi(p) \\
        &= \sum_{p|n}\phi(p)\sum_{pd|n} \mu(d) &\\
        &= \phi(p) 
      \end{align*}
    \end{pf}
  \end{lemma}
  \begin{align*}
      f(n) &= \sum_{i=1}^n\phi(i) \\
      &= \sum_{i=1}^n \sum_{d|n}\mu(d) \frac i d \\
      &= \sum_{d=1}^n \mu(d) \sum_{k=1}^{\lfloor \frac n d \rfloor} k \\
      &= \sum_{d=1}^n \mu(d)\Bigg(\frac 1 2(\frac n d)^2 + O(\frac n d)\Bigg) \\
      &= \frac {n^2} 2 \sum_{d=1}^n \frac {\mu(d)} {d^2}  + O(n\sum_{i=1}^n\frac 1 d) \\
      &= \frac {n^2} 2 \sum_{d=1}^\infty \frac {\mu(d)} {d^2} + O(n^2\sum_{d>n}\frac 1 {d^2}) + O(n\sum_{i=1}^n\frac 1 d) 
  \end{align*}

  得到$k=2$, $f \in \Theta(n^2)$。
\newpage
  (b) 接上题
  $$C = \frac 1 2 \sum_{d=1}^\infty \frac {\mu(d)} {d^2}$$
  我们运用 \textbf{Lemma 2}, 
  \begin{align*}
    1 
    &= \sum_{n=1}^\infty \frac 1 {n^2} \sum_{d|n} \mu(d) \\
    &= \sum_{d=1}^\infty \sum_{m=1}^\infty\frac {\mu(d)} {(dm)^2} \\
    &= \sum_{d=1}^\infty \frac {\mu(d)} {d^2} \sum_{m=1}^\infty \frac 1 {m^2}
  \end{align*}
  所以 $$C = \frac 1 2 \Big( \sum_{m=1}^\infty \frac 1 {m^2} \Big)^{-1}=\frac 3 {\pi^2}$$
\end{sol} 
\end{problem}

\end{document}

