\documentclass[UTF8]{ctexart}
\usepackage{amsmath}
\usepackage{amssymb}
\usepackage{amsthm}
\usepackage{graphicx}
\usepackage{CJK}
\usepackage{float}
\usepackage{mdframed}
\providecommand{\abs}[1]{\lvert#1\rvert}
\providecommand{\norm}[1]{\lVert#1\rVert}
\providecommand{\ud}[1]{\underline{#1}}

\newmdtheoremenv{thm}{Theorem}
\newmdtheoremenv{lemma}[thm]{Lemma}
\newmdtheoremenv{fact}[thm]{Fact}
\newmdtheoremenv{cor}[thm]{Corollary}
\newtheorem{eg}{Example}
\newtheorem{ex}{Exercise}
\newmdtheoremenv{defi}{Definition}
\newenvironment{sol}
  {\par\vspace{3mm}\noindent{\it Solution}.}
  {\qed \\ \medskip}

\newcommand{\expe}{{\mathsf E}}
\newcommand{\pr}{{\mathsf{Pr}}}


\setlength{\parindent}{0pt}
%\setlength{\parskip}{2ex}
\newenvironment{proofof}[1]{\bigskip\noindent{\itshape #1. }}{\hfill$\Box$\medskip}

\theoremstyle{definition}
\newtheorem{problem}{Problem}
\newtheorem*{problem*}{Problem}

\pagenumbering{gobble}

\begin{document}

\title{CS477 Combinatorics: Homework 12}
\author{于峥 518030910437}
\date{\today}

\maketitle

\begin{problem}
    在一个学校有 1000 个俱乐部,每个俱乐部有 11 个人,俱乐
    部之间的成员可能会重复。证明,可以将所有学生分成两组,使得每个俱乐
    部都同时有来自两组的学生。
    \begin{sol}
        我们将俱乐部和学生的关系看成二分图$G = (L \cup R, E)$, $L$代表俱乐部,
        $R$代表学生,如果学生属于俱乐部那么这两个点之间有边相连。那么我们需要证明的是能够做到将$R$二染色后,使得$L$中每个元素的
        邻点集合包含两种颜色。

        我们将$R$中的点以$\frac 1 2$的概率随机染成黑白两色。那么存在俱乐部只有一种颜色
        的邻点的概率
        $$
        P \leq 1000 \cdot 2^{-11} \cdot 2 < 1
        $$
        所以$1-P > 0$, 因此命题得证。
    \end{sol}
\end{problem}

\begin{problem}
    一个竞赛图是一个每两个点之间恰好有一条边的有向图。
    证明:对每个 $n$,存在一个竞赛图包含至少 $n!/2^{n−1}$ 条哈密尔顿路。
    \begin{sol}
        考虑对$K_n$中的边等概率随机定向。那么$K_n$中哈密尔顿路的期望数量为
        $E(P) = \frac {n!} {2^{n-1}}$, 该式的意思是考虑每一个排列为哈密顿尔
        路的概率,因此肯定存在一个竞赛图包含至少 $n!/2^{n−1}$ 条哈密尔顿路。
    \end{sol}
\end{problem}

\begin{problem}
    $G = (V, E)$ 是一个二分图,$|V| = n$,$k > \log_2 n$ 是一个整数。
在 $G$ 的每个顶点 $v$ 上有一个候选的颜色集合 $C(v)$ 满足 $|C(v)| \geq k$。证明:
可以给每个点 $v$ 选一个 $C(v)$ 中的颜色,使得 $G$ 中任意相邻两点不同色。
    \begin{sol}
        题目条件相当于把所有颜色分为两组,对于每一个左边点的$C$含有至少一个第一组的颜色,
        右边点的$C$含有至少一个第二组的颜色。对于任意一个点它被分到两组的概率是$\frac 1 2$,
        那么对于左边每一个颜色集合$C_i$, 如果其中含有第一组的颜色那么$X_i=0$, 否则为$1$。
        对于右边每一个颜色集合$C_i$, 如果其中含有第二组的颜色那么$X_i=0$, 否则为$1$。

        那么$E(\sum X_i)=\frac n {2^k} < 1$, 所以存在合法的分组。对于每一个点,
        如果在左边就任选一个第一组的颜色,在右边就选第二组的颜色即可。
    \end{sol}
\end{problem}

\begin{problem}
    图 $G = ([1000], E)$ 中有至多 27000 个三角形。证明:G 中有
一个 70 点的不含三角形的诱导子图。
    \begin{sol}
        我们考虑$G$中所有的三角形,在$\binom{[1000]}{100}$中有$\binom{997}{97}$个集合
        包含这个三角形。因此所有$\binom{[1000]}{100}$中的元素包含的三角形个数的期望
        $$
        E = \frac {{27000} {\binom{997}{97}}} {\binom{1000}{100}} < 26。5
        $$
        所以$G$ 中有一个 100个点的三角形数量小于等于26的诱导子图, 从每个三角形中删去1个点,
        因此图中一定存在大于等于$74$个点的不含三角形的诱导子图。

    \end{sol}
\end{problem}

\begin{problem}
    $G = (V, E)$ 是一个 n 阶的、20 正则的简单图。
    证明:存在 $W \subseteq V$ ,使得 $V - W$ 中任意一个顶点在 $W$ 中至少有一个邻居,
    并且 $|W| \leq n/5$.

    \begin{sol}
        不难把题目中的条件转化找到$W \subseteq V$, 成对于每一个点,要么它在$W$中,要么它的的某个邻点在$W$中。

        我们将每个点以概率$p$的加入$S$, 那么$E(|S|)=pn$, 而$T$是那些不满足要求的点,即不在
        $S$中也没有邻居在$S$中的点。$X_u$为一随机变量,如果点$u$在$T$中值为$1$,否则为$0$。
        那么$E(|X_u|) \leq (1-p)^{21}$。
        \begin{align*}
            E(|S \cup T|) 
            \leq np + n(1-p)^{21}
        \end{align*}

        因此取$p=1 - {21^{-\frac 1 {20}}}$, 那么$E(|S\cup T|) < \frac n 5$, 因此存在这样的$W$.
    \end{sol}
\end{problem}

\begin{problem}
    \textbf{HW11}
    证明:对任意的$k,l$,都存在图$G$使得$\chi(G) > k, g(G) > l$。
    \begin{sol}
      由于一个图的色数为$k$意味着将它分为了$k$个不相交的独立集,因此不难得到
      $$
      \alpha (G) < \frac {|G|} k \Rightarrow \chi(G) > k
      $$
      假设图$G$上每条边以$p$的概率出现,我们记事件$\{ \alpha (G) \geq \frac {|G|} k \}$
      为$X$, 事件$\{ g(X) \leq l \}$为$Y$。那么我们只需使$P(X) + P(Y) < 1$即可。
    
      首先考虑$P(X)$, 图$G$上每条边以$p$的概率出现,记$r=\frac {n} {k}$, 那么
      $$
      P(X) \leq \binom{n}{r}(1-p)^{\binom{r}{2}} \leq 2^ne^{-p\binom {r} {2}}
      $$
      我们只要让$p \geq \frac n {\binom{r}{2}} = \frac {2k^2} {n - k}$即可。容易看到只要$n$足够大,
      我们就能让$P(X)$小于任何正常数。
    
      再考虑$P(Y)$, 当$n > l$时, $P(Y)$不可能为$1$, 此时无论$P(Y)$为何值,
      我们都能取足够大的$n$,使得$P(X) + P(Y) < 1$, 因此命题成立。
    \end{sol}
    \end{problem}
\end{document}
