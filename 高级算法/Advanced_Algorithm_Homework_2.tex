% This is a template for lecture notes.
\documentclass{article}
\usepackage[UTF8]{ctex}
\usepackage{amssymb}
\usepackage{amsmath}
\usepackage{amsthm}
\usepackage{geometry}
\usepackage{booktabs}
\usepackage{bm}
\usepackage{tcolorbox}
\CTEXoptions[today=old]

%Some commonly used notations
%\geometry{a4paper,bottom = 3cm,left = 3cm, right = 3cm}

%for reference
\usepackage{hyperref}
\usepackage[capitalise]{cleveref}

\newtheorem{theorem}{Theorem}
\newtheorem{lemma}[theorem]{Lemma}
\newtheorem{proposition}[theorem]{Proposition}
\newtheorem{corollary}[theorem]{Corollary}
\newtheorem{fact}[theorem]{Fact}
\newtheorem{definition}[theorem]{Definition}
\newtheorem{remark}[theorem]{Remark}
\newtheorem{question}[theorem]{Question}
\newtheorem{answer}[theorem]{Answer}
\newtheorem{problem}[theorem]{Problem}
\newtheorem{example}[theorem]{Example}
\newenvironment{solution}{\noindent \textit{proof:}}{$\Box$}
\newtheorem{observation}[theorem]{Observation}

%to use newcommand for convenience
\newcommand\field{\mathbb{F}}
\newcommand\Real{\mathbb{R}}
\newcommand\Q{\mathbb{Q}}
\newcommand\Z{\mathbb{Z}}
\newcommand\complex{\mathbb{C}}

%this is how we define operators.
\DeclareMathOperator{\rank}{rank} % rank

\title{Advanced Algorithm Homework 2}
\author{于峥 518030910437}
\date{\today}

\begin{document}
    \maketitle
    
\begin{problem}
\end{problem}
\begin{solution}
    
    The permutation will be a derangement without fix point, we know that
    
    $$
    D_n = n! \sum_{i=2}^n\frac {(-1)^i} {i!}
    $$
    
    So we can calculate the variance, let $X$ be the r.v. of number of fix points.
    \begin{align*}
        \mathrm{E}(X^2) 
        &= \frac 1 {n!} \sum_{i=0}^n i^2 \binom{n}{i}D_{n-i} \\
        &= \sum_{i=1}^n \sum_{j=2}^{n-i} \frac i {(i-1)!} \frac {(-1)^j} {j!} 
    \end{align*}
    
    Let $X_i$ be r.v. whether i-th node is a fix point, then $\mathem{E}(X_i) = \frac 1 n$, so
    $$
    \mathrm{E}(X) = \sum \mathrm{E}(X_i)= 1  = \sum_{i=1}^n \sum_{j=2}^{n-i} \frac 1 {(i-1)!} \frac {(-1)^j} {j!}
    $$
    
    \begin{align*}
        \mathrm{Var}(X) 
        &= \mathrm{E}(X^2) - \mathrm{E}(X)^2 \\
        &= \sum_{i=1}^n \sum_{j=2}^{n-i} \frac 1 {(i-1)!} \frac {(-1)^j} {j!} (i-\mathrm{E}(X)) \\
        &= \sum_{i=2}^n \sum_{j=2}^{n-i} \frac 1 {(i-2)!} \frac {(-1)^j} {j!} \\
        &= \sum_{i=1}^{n-1} \sum_{j=2}^{n-i-1} \frac 1 {(i-1)!} \frac {(-1)^j} {j!} \\
        &= 1
    \end{align*}
    
    At last step, note that the form of
    $$
        \sum_{i=1}^{n-1} \sum_{j=2}^{n-i-1} \frac 1 {(i-1)!} \frac {(-1)^j} {j!}
    $$
    is same as $\mathrm{E}(X)$, so the formula equals $1$. In special, we know $\mathrm{Var}(X)=0$ when $n=1$.
\end{solution}

\begin{problem}
\end{problem}
\begin{solution}
    For the first problem, we can calculate the probability
    \begin{align*}
        \mathrm{Pr}(\text{no two balls in same bin}) 
        &= \prod_{i=1}^{c_1\sqrt{n}-1}\left(1-\frac i n\right) \\
        &\leq \exp (-\sum_{i=1}^{c_1\sqrt{n}-1} \frac i n) \\
        &\leq  e^{\frac {c_1\sqrt{n}-c_1^2n} {2n}} \\
    \end{align*}
    
    So we need $\frac {c_1\sqrt{n}-c_1^2n} {2n} \leq -1$ for all $n$, it means $\frac {c_1} {c_1^2-2} \leq \sqrt{n}$ and $c_1 > \sqrt{n}$. Hence, $c_1 > 2$
    
    For the second constant, 
    \begin{align*}
        \mathrm{Pr}(\text{no two balls in same bin}) 
        &= \prod_{i=1}^{c_2\sqrt{n}-1}\left(1-\frac i n\right) \\
        &\geq \exp (-\sum_{i=1}^{c_2\sqrt{n}-1} \frac i n + \frac {i^2} {n^2}) \\
        &\geq \exp (- \frac {c_2\sqrt{n}(c_2\sqrt{n}-1)} {2n} - \frac {(c_2\sqrt{n}-1)c_2\sqrt{n}(2c_2\sqrt{n}-1)} {6n^2} )  \\
        &\geq \exp (- \frac {c_2^2} {2} )  \\
    \end{align*}
    
    So we need $\frac {c_2^2} {2}  < \frac 1 {\log_2 e} $ for all $n$, so we get
    $0 \leq c_2 \leq \sqrt{2\ln 2}$
\end{solution}

\begin{problem}
\end{problem}
\begin{solution}
    (1) Note $X_i \sim \text{Binom}(n, \frac 1 n)$, so we let $Y_i \sim \text{Poisson}(1)$.
    $\mathrm{Pr}(\bigcap Y_i=1) = \prod \mathrm{Pr}(Y_i=0) = \frac 1 {e^n}$. By Corollary 5.9 we get the upper bound is $\frac {\sqrt{n}} {e^{n-1}}$.
    
    (2) Note that $n$ balls in $n$ different bins is a permutation, so $\mathrm{Pr}(\bigcap X_i=1) = \frac {n!} {n^n}$.
    
    (3) Note that the probability that a Poisson random variable with parameter n takes on the value n is $\frac {e^{-n}n^n} {n!}$, we verify it is two probabilities differ by the multiplicative factor.
    
    From Theorem 5.6, we have 
    $$
    \mathrm{Pr}(\bigcap X_i=1)=\mathrm{Pr}(\bigcap Y_i=1|\sum Y_i = n) = 
    \mathrm{Pr}(\bigcap Y_i=1)/\mathrm{Pr}(\sum Y_i = n)
    $$
    
    and the $\sum Y_i \sim \text{Poisson}(n)$, so we get such result.
\end{solution}

\begin{problem}
\end{problem}
\begin{solution}
    Without general, we assume $0 \leq x_n \leq x_{n-1} \leq \cdots \leq x_1$, and define r.v $X_S = \sum_{i\in S}\epsilon_ix_i$.
    
    We split $x$ into two sets, 
    
    \begin{itemize}
        \item If $n=1$, $$\mathrm{Pr}(x_i \leq 1) = 1$
        \item If $x_1 \leq \frac 1 2$, let $A = [1, k]$ and $B = [k+1, n]$, it holds 
        $$
        \sum_{i=1}^{k-1}x_i^2 < \frac 1 4 ~~~ \text{and} ~~~ \sum_{i=1}^kx_i^2 \geq \frac 1 4
        $$
        
        Note that $\mathrm{Var}(X_S) = \sum_{i\in S}x_i^2$, so $\mathrm{Var}(X_A)\in [\frac 1 4, \frac 1 2]$ and 
         $\mathrm{Var}(X_B) \in [\frac 1 2, \frac 3 4]$, then we using Chebyshev inequality
         ,
         
        $$
        \mathrm{Pr}(|X_A| \geq 1) \leq \mathrm{Var}(X_A) \leq \frac 1 2 ~~~\text{and}~~~
        \mathrm{Pr}(|X_B| \geq 1) \leq \mathrm{Var}(X_B) \leq \frac 3 4
        $$
        
        Let $p=\mathrm{Pr}(\text{the signs of } X_A \text{ and } X_B \text{ is different})$, we know that $p = \frac 1 2$.
        
        So 
        $$
        \mathrm{Pr}(|X_{[n]}| \leq 1) \geq p\mathrm{Pr}(|X_A| < 1)\mathrm{Pr}(|X_B| < 1) \geq \frac 1 {16}
        $$
        
        \item If $x_1 > \frac 1 2$, let $A = [1], B = [2, n]$. Thereforce,
        
        $$
        \mathrm{Pr}(|X_A| \leq 1) = 1 ~~~ \text{and} ~~~ \mathrm{Pr}(|X_B| \geq 1) \leq \mathrm{Var}(X_B) \leq \frac 3 4
        $$
        
        And 
        $$
        \mathrm{Pr}(|X_{[n]}| \leq 1) \geq p\mathrm{Pr}(|X_A| < 1)\mathrm{Pr}(|X_B| < 1) \geq \frac 1 {8}
        $$
    \end{itemize}
    
    Hence, we can choose $c = \frac 1 {16}. $
\end{solution}

\end{document} 
