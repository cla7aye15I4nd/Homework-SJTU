% This is a template for lecture notes.
\documentclass{article}
\usepackage[UTF8]{ctex}
\usepackage{amssymb}
\usepackage{amsmath}
\usepackage{amsthm}
\usepackage{geometry}
\usepackage{booktabs}
\usepackage{bm}
\usepackage{tcolorbox}
\usepackage{enumerate}

\CTEXoptions[today=old]

%Some commonly used notations
%\geometry{a4paper,bottom = 3cm,left = 3cm, right = 3cm}

%for reference
\usepackage{hyperref}
\usepackage[capitalise]{cleveref}

\newtheorem{theorem}{Theorem}
\newtheorem{lemma}[theorem]{Lemma}
\newtheorem{proposition}[theorem]{Proposition}
\newtheorem{corollary}[theorem]{Corollary}
\newtheorem{fact}[theorem]{Fact}
\newtheorem{definition}[theorem]{Definition}
\newtheorem{remark}[theorem]{Remark}
\newtheorem{question}[theorem]{Question}
\newtheorem{answer}[theorem]{Answer}
\newtheorem{problem}[theorem]{Problem}
\newtheorem{example}[theorem]{Example}
\newenvironment{solution}{\noindent \textit{proof:}}{$\Box$}
\newtheorem{observation}[theorem]{Observation}

%to use newcommand for convenience
\newcommand\field{\mathbb{F}}
\newcommand\Real{\mathbb{R}}
\newcommand\Q{\mathbb{Q}}
\newcommand\Z{\mathbb{Z}}
\newcommand\complex{\mathbb{C}}

%this is how we define operators.
\DeclareMathOperator{\rank}{rank} % rank

\title{Advanced Algorithm Homework 5}
\author{于峥 518030910437}
\date{\today}

\begin{document}
    \maketitle
    
\begin{problem}
\end{problem}
\begin{solution}
    Construct three r.v. $a, b, c$ are uniformly random in $[0, 3]$, and $a+b+c=6$,
    then event $X_v = [v \leq 2]$. We can know that 
    $$ \mathrm{Pr}[X_a] = \mathrm{Pr}[X_b] = \mathrm{Pr}[X_c] = 2/3 $$
    $$ \mathrm{Pr}[X_a \land X_b] = \mathrm{Pr}[X_a \land X_c] = \mathrm{Pr}[X_b\land X_c] = 4/9 $$
    $$ \mathrm{Pr}[X_a\land X_b \land X_c]= \mathrm{Pr}[\overline{X_a}\land \overline{X_b} \land \overline{X_c}]=0 ~~ (*)$$
    So $X_a, X_b, X_c$ are pairwise independent but not mutually independent, it means the dependency graph is three
    isolated point. We set $x(X_a)=x(X_b)=x(X_c)=\frac 5 6$, it is easily to verify the function satisfy 
    asymmetric Lovász local lemma condition, then we get
    $$
    \mathrm{Pr}[\overline{X_a}\land \overline{X_b} \land \overline{X_c}] \geq \frac {1} {6^3}
    $$
    which contradict to $(*)$. 
    Hence, the requirement that $A$ and $A\backslash \{N^+(A)\}$ are mutually independent cannot be weaken to pairwise independent.
\end{solution}

\begin{problem}
\end{problem}
\begin{solution}
    Let $C_u = \{ C \text{ is a clause } | u \in C \}$, then we can induce on size of $C_u$.
    \paragraph{Base Step} $C_u = \emptyset$. 
    $\sigma$ be a random assignment, 
    $\varphi|_\sigma = \mathtt{true}$ means $\sigma$ is a satisfying assignment of $\varphi$. We easily know that
    \begin{align*}
        \mathrm{Pr}_{\mathcal{D}}[u = \mathtt{true}] 
        = \mathrm{Pr}[u = \mathtt{true}|\varphi|_\sigma = \mathtt{true}] = \frac 1 2
    \end{align*}
    
    \paragraph{Induction Step} $|C_u| \geq 1$. Let $\varphi' = \varphi \backslash C(C \in C_u)$, then
    \begin{align*}
        \mathrm{Pr}_{\mathcal{D}}[u = \mathtt{true}] 
        &= \mathrm{Pr}[u = \mathtt{true}|\varphi|_\sigma = \mathtt{true}] \\
        &= \frac {\mathrm{Pr}[u = \mathtt{true} \land \varphi|_\sigma = \mathtt{true}]} {\mathrm{Pr}[\varphi|_\sigma = \mathtt{true}]} \\
        &\leq \frac {\mathrm{Pr}[u = \mathtt{true} \land \varphi'|_\sigma = \mathtt{true}]} {\mathrm{Pr}[\varphi|_\sigma = \mathtt{true}|\varphi'|_\sigma = \mathtt{true}]\mathrm{Pr}[\varphi'|_\sigma = \mathtt{true}]} \\
        &\leq \frac {\mathrm{Pr}[u = \mathtt{true} \land \varphi'|_\sigma = \mathtt{true}]} {(1-x(C))\mathrm{Pr}[\varphi'|_\sigma = \mathtt{true}]} &(*)\\
        &\leq \frac {\mathrm{Pr}[u = \mathtt{true} | \varphi'|_\sigma = \mathtt{true}]} {1-x(C)} \\
        &\leq \frac 1 2 \prod_{u\in C}(1-x(C))^{-1}
    \end{align*}
    Step $(*)$ use the lemma proofed in class. If $x$ satisfies the 
    Lovasz Local Lemma condition, then 
    $$
    \forall i \not\in S, \mathrm{Pr}[A_i|\bigcap_{i\in S}\overline{A_i}] \leq x(A_i)
    $$
    It means 
    $$\mathrm{Pr}[\varphi|_\sigma = \mathtt{true}|\varphi'|_\sigma = \mathtt{true}] = \mathrm{Pr}[C|_{\sigma}=\mathtt{true}|\varphi'|_\sigma = \mathtt{true}]\geq 1-x(C)
    $$
    Hence, we finished the proof.
\end{solution}

\begin{problem}
\end{problem}
\begin{solution}
    (1) The counterexample is $P_3(x-y-z)$. There are 5 possible kinds independent set $\emptyset, \{x\}, \{y\}, \{z\}, \{x,z\}$. We consider the probability of the result is $\{x,z\}$. Let $X$ be the event of result is$y$. We can simply calculate it.
    $$
    \mathrm{Pr}[X] = \frac 1 8 + \frac 1 8 \cdot \frac 1 3 \not= \frac 1 5
    $$
    So the algorithm is not uniform.
\end{solution}

\end{document} 
