% This is a template for lecture notes.
\documentclass{article}
\usepackage[UTF8]{ctex}
\usepackage{amssymb}
\usepackage{amsmath}
\usepackage{amsthm}
\usepackage{geometry}
\usepackage{booktabs}
\usepackage{bm}
\usepackage{tcolorbox}
\CTEXoptions[today=old]

%Some commonly used notations
%\geometry{a4paper,bottom = 3cm,left = 3cm, right = 3cm}

%for reference
\usepackage{hyperref}
\usepackage[capitalise]{cleveref}

\newtheorem{theorem}{Theorem}
\newtheorem{lemma}[theorem]{Lemma}
\newtheorem{proposition}[theorem]{Proposition}
\newtheorem{corollary}[theorem]{Corollary}
\newtheorem{fact}[theorem]{Fact}
\newtheorem{definition}[theorem]{Definition}
\newtheorem{remark}[theorem]{Remark}
\newtheorem{question}[theorem]{Question}
\newtheorem{answer}[theorem]{Answer}
\newtheorem{problem}[theorem]{Problem}
\newtheorem{example}[theorem]{Example}
\newenvironment{solution}{\noindent \textit{proof:}}{$\Box$}
\newtheorem{observation}[theorem]{Observation}

%to use newcommand for convenience
\newcommand\field{\mathbb{F}}
\newcommand\Real{\mathbb{R}}
\newcommand\Q{\mathbb{Q}}
\newcommand\Z{\mathbb{Z}}
\newcommand\complex{\mathbb{C}}

%this is how we define operators.
\DeclareMathOperator{\rank}{rank} % rank

\title{Advanced Algorithm Homework 1}
\author{于峥 518030910437}
\date{\today}

\begin{document}
    \maketitle
    
\begin{problem}
\end{problem}
\begin{solution}
    (1)  Let $Y_i$ be the r.v. if the $i$th position is the start of one streak of length $\log_2n+1$ and $Y$ is the number of streaks $\log_2n+1$.
    
        $$
        \mathrm{E}(Y) = \sum_{i=1}^{n - \log_2n} \mathrm{E}(Y_i) = \sum_{i=1}^{n - \log_2n} 2^{-\log_2n} = 1 - o(1)
        $$ 
    
    (2) $m = \lfloor \log_2n - 2\log_2\log_2 n \rfloor$. Break the sequence of flips up into disjoint blocks
    of $m$ consecutive. So the number of blocks will less than $n/\log_2n-1$
    
    Let $Y$ be the event there is no streak of length at least $m$ is less than $1/n$ and
    $X$ be the event any blocks is not a streak. Note that $X \subset Y$, Hence
    $$
        \mathrm{Pr}(Y) \le \mathrm{Pr}(X)
    $$
    
    then we bound it
    \begin{align*}
        \mathrm{Pr}(X) 
        &\leq \left( 1-2^{1-m}\right)^{n/m-1} \\
        &\leq \left( 1-2^{2\log_2\log_2n - \log_2n}\right)^{n/m-1} \\
        &\leq \left(1-\frac {\log_2^2 n} {n}\right)^{n/\log_2n-1} \\
        &\leq e^{\frac {\log_2^2n} {n} - \log_2n} \\
        &= n^{\log_2 e (\frac {\log_2 n} {n} - 1)}
    \end{align*}
    
    So we just need  proof $\log_2 e (\frac {\log_2 n} {n} - 1) \leq -1$ when n is large,
    it means 
    $$ \log_2e > 1 \geq \frac {n} {n-\log_2 n} $$
    
    Obviously, the formula is satisfied when $n$ is large enough.
    
    
\end{solution}

\begin{problem}
\end{problem}
\begin{solution}
    Let  $X$ be the r.v. of number of rolls until one sixes appears, $Y$ be the r.v. of number of rolls until the first pair of consecutive sixes appears, it is easy to get $\mathrm{E}(X) = 6$. Hence,
    
    $$\mathrm{E}(Y) = \mathrm{E}(X) + \mathrm{E}(\frac 1 6 + \frac 5 6(1 + Y)) \Rightarrow \mathrm{E}(Y) = 42$$
\end{solution}

\begin{problem}
\end{problem}
\begin{solution}
    A permutation can be represented as set of cycles, so consider insert 1 to $n$ to a empty graph. We have two choice,
    \begin{itemize}
        \item As a self-loop, there will be new cycle.
        \item Become the successor of a node that already exists in graph.
    \end{itemize}
    
    Let $X_i$ be the r.v., the number of cycle increased after insert the $i$, then $\mathrm{E}(X_i) = \frac 1 i$
    
    $X$ be the number of cycles in permutation, 
    $\mathrm{E}(X) = \sum_{i=1}^n\mathrm{E}(X_i) = \sum_{i=1}^n \frac 1 i = H_n$.
\end{solution}

\begin{problem}
\end{problem}
\begin{solution}
    Enumerate the number of HEADs to calculate the expectation of $|a-b|$,
    \begin{align*}
        \frac 1 {2^n}\sum_{i=0}^n|n-2i|\binom{n}{i}
        &=\frac 1 {2^{n-1}}\sum_{i=0}^{\lfloor\frac n 2\rfloor}(n-2i)\binom{n}{i} \\
        &=\frac n {2^{n-1}}(1+\sum_{i=1}^{\lfloor\frac n 2\rfloor}\binom{n}{i} - 2\binom{n-1}{i-1})\\
        &=\frac n {2^{n-1}}(1+\sum_{i=1}^{\lfloor\frac n 2\rfloor}\binom{n-1}{i} - \binom{n-1}{i-1})\\
        &=\frac n {2^{n-1}}\binom{n-1}{\lfloor\frac n 2\rfloor}
  \end{align*}  
  
  It is easy to see $\frac n {2^{n-1}} = \Theta(n / 2^n)$, and note that $\frac 1 2 \binom{n}{\lfloor n / 2 \rfloor} \leq \binom{n-1}{\lfloor n / 2 \rfloor} \leq \binom{n}{\lfloor n / 2 \rfloor}$, so we can 
  just estimate the $\binom{n}{\lfloor n / 2 \rfloor}$. When $n$ is even,
  \begin{align*}
    \frac n {2^{n-1}}\binom{n-1}{n/2} 
    &= \Theta \left(\frac{n}{2^n}\right) \Theta \left( \binom{n}{n/2} \right) \\
    &= \Theta \left(\frac{n}{2^n}\right) \Theta \left(n!/(\frac {n}{2}!)^2\right) \\
    &= \Theta \left(\frac{n}{2^n}\right) \Theta \left(\frac {\sqrt{n}(n/e)^n} {n (n/(2e))^n}\right) \\
    &= \Theta (\sqrt{n})
  \end{align*}
  
  When $n$ is odd,
  \begin{align*}
    \frac n {2^{n-1}}\binom{n-1}{\frac {n-1} 2} 
    &= \Theta \left(\frac{n}{2^n}\right) \Theta \left((n-1)!/(\frac {n-1}{2}!)^2\right) \\
    &= \Theta \left(\frac{n}{2^n}\right) \Theta \left(\frac {\sqrt{n-1}((n-1)/e)^{n-1}} {(n-1) ((n-1)/(2e))^{n-1}}\right) \\
    &= \Theta (\sqrt{n})
  \end{align*}
  
  \textbf{Anthoer way}
  
  Define the r.v. $X_i$, If the direction of the i-th coin occuers more, the $X_i = +1$, if less, the $X_i = -1$. Hence. If equal, the $X_i = 0$, $|a-b|=\sum X_i$
  
  When $|a-b|\not=0$ we reverse the i-th coin, the contribution will  become to $-X_i$
  if and only if the $|a-b| > 2$. 
  
  If $n$ is odd, a coin will have contribution only when $|a-b|=1$, 
  so $\mathrm{E}(X_i)=2\binom{n}{\frac {n-1}2}/2^{n}=\binom{n+1}{\frac {n+1} 2} / 2^n$.
  $n$ is even is same, $\mathrm{E}(X_i)= \binom{n}{n/2}/2^n$.
  
  Using the Stirling's formula, we can know that $\mathrm{E}(X_i)=\Theta(\frac 1 {\sqrt{n}})$, so 
  $\mathrm{E}(|a-b|)=\Theta(\sqrt{n})$.
\end{solution}

\end{document} 
