\documentclass{ximera}

\usepackage[UTF8]{ctex}

\title{Mathematical Logic}
\author{于峥}

\begin{document}
\begin{abstract}
    homework 2
\end{abstract}
\maketitle

\begin{problem}
    Prove that:
    
    (a) If A and B are both groups, then so is A $\times$ B.

    (b) If A and B are both equivalence relations, then so is A $\times$ B.

    (c) For two fields A and B, their direct product A $\times$ B is not necessarily a field

    \begin{solution} .

        (a) 
        $a_0, a_1, a_2 \in A, b_0, b_1, b_3 \in B$,$\epsilon_a, \epsilon_b$ 分别为$A,B$中的单位元。
        \begin{itemize} 
            \item \textbf{封闭性}: 因为 $a_0 \cdot a_1 \in A$, $b_0 \cdot b_1 \in B$, 所以 $(a_0, b_0) \cdot (a_1, b_1) = (a_0 \cdot a_1, b_0 \cdot b_1) \in A \times B$。
            \item \textbf{结合律}:
            $$
            \begin{aligned}
                ((a_0, b_0) \cdot (a_1, b_1)) \cdot (a_2, b_2) 
                &=  ((a_0 \cdot a_1) \cdot a_2, (b_0 \cdot b_1) \cdot b_2)\\
                &=  (a_0 \cdot (a_1 \cdot a_2), b_0 \cdot (b_1 \cdot b_2))\\
                &=  (a_0, b_0) \cdot ((a_1, b_1) \cdot (a_2, b_2))
            \end{aligned}
            $$
            \item \textbf{单位元}:易验证$(\epsilon_a, \epsilon_b)$为单位元。
            \item \textbf{逆元元}:易验证$(a_0^{-1}, b_0^{-1})$为$(a_0, b_0)$单位元。
        \end{itemize}
        (b)
        \begin{itemize}
            \item \textbf{传递性}: 
            $(a_0, b_0) \sim (a_1, b_1) \land (a_1, b_1) \sim (a_2, b_2) \Rightarrow a_0 \sim a_2 \land b_0 \sim b_2 \Rightarrow (a_0, b_0) \sim (a_2, b_2)$。
            \item \textbf{自反性}:$a_0 \sim a_0 \land b_0 \sim b_0 \Rightarrow (a_0, b_0) \sim (a_0, b_0)$
            \item \textbf{对称性}:$(a_0, b_0) \sim (a_1, b_1) \Rightarrow a_0 \sim a_1 \land b_0 \sim b_1 \land a_1 \sim a_0 \land b_1 \sim b_0 \Rightarrow (a_0, b_0) \sim (a_1, b_1)$
        \end{itemize}
        (c) 
        例如$A$,$B$都为$R$, $(a, 0)$ 没有乘法逆元$a \neq 0$。
    \end{solution}
\end{problem}

\begin{problem} Prove Example 1.12.
    \begin{solution}
    
       首先 $ \forall v_0\exists v_0 v_0 \circ v_1 \equiv e $,
       又 $\exists v_2, v_1 \circ v_2 \equiv e$。

       所以  $\forall v_0\exists v_1$
       $$
       \begin{aligned}  
            v_1 \circ v_0 
            &\equiv (v_1 \circ v_0) \circ e \\
            &\equiv (v_1 \circ v_0) \circ (v_1 \circ v_2) \\
            &\equiv v_1 \circ (v_0 \circ v_1) \circ v_2 \\
            &\equiv v_1 \circ e \circ v_2 \\
            &\equiv  v_1 \circ v_2 \\
            &\equiv e
       \end{aligned}
       $$
       有可以推出
       $$e\circ v_0 \equiv (v_0 \circ v_1) \circ v_0 \equiv v_0 \circ (v_1 \circ v_0) \equiv v_0 \circ r \equiv v_0$$
    \end{solution}
\end{problem}

\begin{problem} An S-formula is positive if it contains no logic symbols $\neg$, $\rightarrow$, and $\leftrightarrow$. Prove that
    every positive formula is satisfiable.
    \begin{solution}
        For a $\mathfrak{A}$, Let $A = \{\epsilon\}$,for a positive $\phi$。
        \begin{itemize}
            \item $\phi = t_1 \equiv t_2$. we have 
                $$
                \begin{aligned}
                    \mathfrak{A} \models \epsilon \equiv \epsilon \Leftrightarrow& 
                                \mathfrak{A}(t_1) \equiv \mathfrak{A}(t_2) \\
                                \Leftrightarrow& t_1 \equiv t_2
                \end{aligned}
                $$
            \item $\phi = Rt_1\dots t_n$, Then we just need Let all relations can be satisfiable.
            \item $\phi = \psi \lor \mathcal{X}$, 
            $$
                \mathfrak{A} \models \psi \land \mathfrak{A} \models \mathcal{X} \Rightarrow \mathfrak{A} \models \psi \land \mathcal{X}
            $$
            \item $\phi = \exists x \psi$, $x$ can just can be $\epsilon$, so $\mathfrak{A}\frac \epsilon x \models \phi$
        \end{itemize}
    \end{solution}
\end{problem}

\begin{problem}
        S-structures $\mathfrak{A}, \mathfrak{B},\mathfrak{C}$

        (1) $\mathfrak{A} \approxeq \mathfrak{A}$

        (2) $\mathfrak{A} \approxeq \mathfrak{B}$  implies  $\mathfrak{B} \approxeq \mathfrak{A}$

        (3) $\mathfrak{A} \approxeq \mathfrak{B}$ and $\mathfrak{B} \approxeq \mathfrak{C}$ then $\mathfrak{A} \approxeq \mathfrak{C}$
        \begin{solution}
            (1) A mapping $\pi$ : $A \rightarrow A$。
            
            Because $\forall x \in A, \pi(x) = x$, so 
            \begin{itemize}
                \item $\pi$ is  bijection.
                \item For any n-ary relation symbol $R \in S$ and $a_0, a_1, \dots a_{n-1} \in A$, 
                $$(a_0 \dots a_{n-1}) \in R^{\mathfrak{A}}\Rightarrow (\pi(a_0) \dots \pi(a_{n-1})=(a_0 \dots a_{n-1}) \in R^{\mathfrak{A}}$$
                \item For any n-ary function symbol $f \in S$ and $a_0, a_1, \dots a_{n-1} \in A$, 
                
                $$\pi (f^{\mathfrak{A}}(a_0 \dots a_{n-1})) = f^{\mathfrak{A}}(\pi(a_0) \dots \pi(a_{n-1})) 
                                                                = f^{\mathfrak{A}}(a_0 \dots a_{n-1})$$
            \end{itemize}

            (2) A mapping $\pi$ : $A \rightarrow B$, Because $\pi$ is bijection, so $\exists \pi^{-1}$ : $B \rightarrow A$ is bijection.
            \begin{itemize}
            \item For any n-ary relation symbol $R \in S$ and $\pi(a_0)=b_0 \dots \pi(a_{n-1})=b_{n-1} \in B$.
            $$
            \begin{aligned}
                (b_0 \dots b_{n-1}) \in R^{\mathfrak{B}} &\Rightarrow (\pi(a_0) \dots \pi(a_{n-1}) \in R^{\mathfrak{B}}\\
            \end{aligned}
            $$
            Because $\pi$ is  bijection, so $(a_0 \dots a_{n-1}) \in R^{\mathfrak{A}}$
            \item For any n-ary function symaol $f \in S$ and $\pi(a_0)=b_0 \dots \pi(a_{n-1})=b_{n-1} \in B$.
            
            $$\pi (f^{\mathfrak{A}}(a_0 \dots a_{n-1})) = f^{\mathfrak{B}}(\pi(a_0) \dots \pi(a_{n-1})) $$
            so
            $$\pi^{-1} (f^{\mathfrak{B}}(b_0 \dots b_{n-1})) = f^{\mathfrak{A}}(\pi^{-1}(b_0) \dots \pi^{-1}(b_{n-1})) $$
            \end{itemize}

            (3) A mapping $\sigma$ : $A \rightarrow B$, $\tau$ : $B \rightarrow C$, $\pi=\tau\sigma$ is bijection.
            \begin{itemize}
                \item  For any n-ary relation symbol $R \in S$,
                $$
                \begin{aligned}
                    (a_0 \dots a_{n-1}) \in R^{\mathfrak{A}} &\Rightarrow (\sigma(a_0) \dots \sigma(a_{n-1})\in R^{\mathfrak{B}}\\
                                                            &\Rightarrow (\tau(\sigma(a_0)) \dots \tau(\sigma(a_{n-1}))\in R^{\mathfrak{C}}\\
                                                            &\Rightarrow (\pi(a_0) \dots \pi(a_{n-1})\in R^{\mathfrak{C}}\\
                \end{aligned}
                $$
                \item For any n-ary function symaol $f \in S$,
                $$
                \begin{aligned}
                    \pi (f^{\mathfrak{A}}(a_0 \dots a_{n-1})) &= \tau(f^{\mathfrak{B}}(\sigma(a_0) \dots \sigma(a_{n-1}))) \\
                        &= f^{\mathfrak{C}}(\pi(a_0) \dots \pi(a_{n-1})) \\
                \end{aligned}
                $$
            \end{itemize}
        \end{solution}
\end{problem}

\begin{problem} Let $\phi, \psi$, and $\mathcal{X}$ be S-formulas. Prove that:

    (a) $(\phi \lor \psi)  \models \mathcal{X}$ if and only if $\phi \models \mathcal{X}$ and $\psi \models \mathcal{X}$.

    (b) $\models \phi \rightarrow \psi$ if and only if $\phi \models \psi$.
    \begin{solution}
        (a) If $\mathfrak{I} \models (\phi \lor \psi)$ and $\mathfrak{I} \models \mathcal{X}$, 
         then $\mathfrak{I} \models \phi$ and $\mathfrak{I} \models \psi$, so $\phi \models \mathcal{X}$ and $\psi \models \mathcal{X}$。
        Similarly, if $\mathfrak{I} \models \phi$, $\mathfrak{I} \models \psi$, $\mathfrak{I} \models \mathcal{X}$, $(\phi \lor \psi) \models \mathcal{X}$。

        (b) If $\mathfrak{I} \models \phi \rightarrow \psi$, then $\mathfrak{I} \models \phi$ implies $\mathfrak{I} \models \psi$,so $\phi \models \psi$,and vice versa。
    \end{solution}
\end{problem}

\begin{problem} Let $S$ be finite, i.e., containing finitely many relation symbols, function symbols, and
    constants. Prove that two finite structures $\mathfrak{A}$ and $\mathfrak{B}$ are isomorphic if and only if for any S-sentence
    $\phi$

    $$\mathfrak{A} \models \phi \Leftrightarrow \mathfrak{B} \models \phi$$
    
    \begin{solution}
        (1) 如果$\mathfrak{A}$ 和 $\mathfrak{B}$ 同构,根据 Lemma 1.5 (The Isomorphism Lemma)这是对的。

        (2) 若$\forall \phi$
        $$\mathfrak{A} \models \phi \Leftrightarrow \mathfrak{B} \models \phi$$

        假设$A = \{x_0, x_1, \dots ,x_{n-1}\}$,

        我们可以找到这样一个命题$\psi$, 这个命题能够描述所有的命题,并且构造了一个双射$\pi$, 根据之前的构造,其他两个条件是显然的。所以$\mathfrak{A}$和$\mathfrak{B}$是同构的。

        
    \end{solution}
\end{problem}

\end{document}