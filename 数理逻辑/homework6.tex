\documentclass{ximera}

\usepackage[UTF8]{ctex}

\begin{document}
\title{Mathematical Logic}
\author{于峥}

\begin{abstract}
    homework 6
\end{abstract}
\maketitle

\begin{problem} Let $\Phi \subseteq L^S$ be finite, and let $ϕ \in L^S$ with $\Phi \vdash \varphi$. Note that a proof might use formulas built on any symbol in S.
    
    Define $S_0 \in S$ to be the set of symbols that occur in $\Phi$ and $\varphi$. Then there is a proof for $\Phi \vdash \varphi$
    such that every formula occurs in the proof is an $S_0$-formula.
    \begin{solution}
        By theorem 1.2, we have a S-interpretation $\mathfrak{I}^{\Phi}$.
        $$
            \mathfrak{I}^{\Phi} \models \varphi \Longleftrightarrow \Phi \vdash \varphi
        $$
        
        same as $S_0$-interpretation $\mathfrak{I}^{\Phi}_0$, let $\mathfrak{I}^{\Phi}_0$ has same interpretation on function and relation with $\mathfrak{I}^{\Phi}$.
        By Coincidenc lemma
        $$
            \mathfrak{I}^{\Phi} \models \varphi \Longrightarrow \mathfrak{I}^{\Phi}_0  \models \varphi \Longrightarrow \Phi \vdash \varphi
        $$
    \end{solution}
\end{problem}

\begin{problem}
    Assume that for every set $A$ there is a well order $\leqslant \subseteq A \times A$. 
    Prove Zorn’s Lemma.
    \begin{solution}
        % Consider the Axiom of Choice:

        % For every non-emptyset $X$, there is a choice function $f : \mathcal{P} / \{ \emptyset \} \rightarrow X$. 
        % such that $f(A) \in A$ for each $A$.

        % Because $A$ there is a well order $\leqslant \subseteq A \times A$,  so we let $f(A)$ 
        % be the minimal element of $A$ for each $A$ in $\mathcal{P}(A) / \{ \emptyset \}$. 
        
        First we proof in any partially ordered set $(S, \leqslant)$
        there is a maximal chain(a chain $C$ for which no $C \cup \{s\}$ is a chain for any s in $S / C$:
        
        For one chain $C_0$, if for any s in $S / C_0$ that $C_0 \cup \{s\}$ is not a chain, then $C_0$ is 
        maximal chain. Otherwise, let $C_1 = C_0 \cup \{s\}$, $C_0 \cup \{s\}$ is a chain. Similarly we can define 
        $C_2, C_3, \cdots, C_n$. and define $M = \{ C_0, C_1, C_2, \cdots, C_n \}$, 
        Because $C_0 \subseteq C_1 \subseteq C_2 \cdots \subseteq C_n$, so $M$ is well-ordering.
        and $M$ is transfinite induction object by transfinite induction.
        
        So $C_n$ is the upperbound of $M$ and the maximal chain of $S$, and $C_n$ is also well-ordering.
        Let $a$ is the upperbound of $C_n$, if $\exists b, a \leq b$, then $C_n \cup \{ b \} \subseteq C_n$.
        hence $a = b$. So $a$ is the maximal element of $S$.
    \end{solution}
\end{problem}

\end{document}
