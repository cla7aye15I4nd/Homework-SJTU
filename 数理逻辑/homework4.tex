\documentclass{ximera}

\usepackage[UTF8]{ctex}

\begin{document}
\title{Mathematical Logic}
\author{于峥}

\begin{abstract}
    homework 4
\end{abstract}
\maketitle

\begin{problem}
    Can you derive the rule of contradiction from the modified contradiction?
    \begin{solution}
        $$
            \begin{aligned}
                1. &\Gamma ~ &\neg\varphi ~~ &\psi      &\text{(premise)}\\
                2. &\Gamma ~ &\neg\varphi ~~ &\neg\psi  &\text{(premise)}\\
                3. &\Gamma ~ &\neg\varphi ~~ &\varphi   &\text{(modified contradiction by 1 and 2)}\\
                4. &\Gamma ~ &\varphi     ~~ &\varphi   &\text{(assumption)}\\
                5. &       ~ &\Gamma      ~~ &\varphi   &\text{(modified contradiction by 3 and 4)}
            \end{aligned}
        $$
    \end{solution}
\end{problem}

\begin{problem} Prove:
    (a) \begin{tabular}{ll}
        $\Gamma$ & $\varphi$\\
        \hline
        $\Gamma$ & $\neg\neg\varphi$\\
        \end{tabular}
    (b) \begin{tabular}{ll}
        $\Gamma$ & $\neg\neg\varphi$\\
        \hline
        $\Gamma$ & $\varphi$\\
        \end{tabular}
    \begin{solution} 
        (a) 
        $$
            \begin{aligned}
                1. &\Gamma &                ~~&\varphi         &\text{(premise)}\\
                2. &\Gamma &\neg\varphi     ~~&\varphi         &\text{(assumption)}\\
                3. &\Gamma &\neg\varphi     ~~&\neg\varphi     &\text{(assumption)}\\
                4. &\Gamma &\neg\varphi     ~~&\neg\neg\varphi &\text{(modified contradiction by 2 and 3)}\\
                5. &\Gamma &\neg\neg\varphi ~~&\neg\neg\varphi &\text{(assumption)}\\
                6. &\Gamma &                ~~&\neg\neg\varphi &\text{(case analysis by 4 and 5)}
            \end{aligned}
        $$
        (b)
        $$
            \begin{aligned}
                1. &\Gamma &            ~~&\neg\neg\varphi         &\text{(premise)}\\
                2. &\Gamma &\neg\varphi ~~&\neg\neg\varphi         &\text{(assumption)}\\
                3. &\Gamma &\neg\varphi ~~&\neg\varphi             &\text{(assumption)}\\
                4. &\Gamma &\neg\varphi ~~&\varphi                 &\text{(modified contradiction by 2 and 3)}\\
                5. &\Gamma &\varphi     ~~&\varphi                 &\text{(assumption)}\\
                6. &\Gamma &            ~~&\varphi                 &\text{(case analysis by 4 and 5)}                 
            \end{aligned}
        $$

    \end{solution}
\end{problem}

\begin{problem} Is the following derivable?
    $$
    \frac {} {\Gamma ~~ \exists x \varphi ~~ \forall x \varphi}
    $$
    \begin{solution}
        We let 
        $$\mathfrak{I}=\{\{0^{\mathfrak{I}}, 1^{\mathfrak{I}}\}, R^{\mathfrak{I}}, \beta\}$$ 
        and $0^{\mathfrak{I}} \in R^{\mathfrak{I}}$,$1^{\mathfrak{I}} \not\in R^{\mathfrak{I}}$.
        if $\varphi := Rx$.

        Then $\mathfrak{I}\frac 0 {v_0} \frac 1 {v_1} \models \Gamma \land \exists x \varphi$.
        $\mathfrak{I}\frac 0 {v_0} \frac 1 {v_1} \not\models \forall x \varphi$.

        So $\{\Gamma ~ \exists x \varphi\} \not\models \forall x \varphi$, it is not correct, so it is not derivable.
    \end{solution}
\end{problem}

\begin{problem}
    Let $S = \{R\}$ with unary relation symbol $R$. 
    $$
        \Phi = \{\exists x Rx \} \cup \{ \neg Ry | \text{for every variable y} \}
    $$
    
    Prove that:

        – $\Phi$ is consistent.

        – There is no term $t \in T^S$, with $\Phi \vdash Rt$.
    \begin{solution}
        (1)
        We construct 
        $$\mathfrak{I} = \{ \{ 0^{\mathfrak{I}}, 1^{\mathfrak{I}} \}, R^{\mathfrak{I}}, \beta \}$$

        where $\beta(v_i)=1^{\mathfrak{I}}$, and $0^{\mathfrak{I}} \in R^{\mathfrak{I}}$, $1^{\mathfrak{I}} \not\in R^{\mathfrak{I}}$。

        (i) because $0^{\mathfrak{I}} \in R^{\mathfrak{I}}$, so $\exists x Rx$ is satisfiable.

        (ii) and $\beta(v_i)=1 \not\in R^{\mathfrak{I}}$, so $\{ \neg Ry | \text{for every variable y} \}$ is satisfiable.

        $\Phi$ is satisfiable $\Rightarrow \Phi$ is consistent (\textbf{Lemma 2.5}).

        (2)  We construct 
        $$\mathfrak{I} = \{ \{ 1^{\mathfrak{I}} \}, R^{\mathfrak{I}}, \beta \}$$

        Where $\beta(v_i)=1^{\mathfrak{I}}$, and $1^{\mathfrak{I}} \not\in R^{\mathfrak{I}}$.
        Because $\{1\}$ is universe.

        So no term $t \in T^{S}$, with $\Phi \vdash Rt$. It is satisfiable.

    \end{solution}
\end{problem}

\begin{problem}
    Let $S = \{R\}$ with unary relation symbol $R$. and
    $$
        \Phi = \{Rx \lor Ry\}
    $$
    
    Prove that:

        – $\Phi$ is consistent.

        – $\Phi \not\vdash Rx$, $\Phi \not\vdash Ry$.

        - $\mathfrak{I}^{\Phi} \not\models \Phi$
    \begin{solution}
        (1) Because $\Phi \not\vdash Rx \land Ry$, so $\Phi$ is not inconsistent. 

        (2) Construct 
        $$
            \mathfrak{I} = \{\{0^{\mathfrak{I}},1^{\mathfrak{I}}\}, R^{\mathfrak{I}}, \beta\}\}
        $$
        and  $0^{\mathfrak{I}} \in R^{\mathfrak{I}}$, $1^{\mathfrak{I}} \not\in R^{\mathfrak{I}}$.

        Then $\mathfrak{I}\frac 0 x \frac 1 y \models \Phi$, and $\mathfrak{I}\frac 0 x \frac 1 y \not\models Ry$.

        $\Phi \not\models Ry \Rightarrow \Phi \vdash Ry \textbf{ is not correct}$, so $\Phi \not\vdash Ry$.

        same as $\Phi \not\vdash Rx$.

        (3) Because $\Phi \not\vdash Rx$, so does not exist $\overline{t} \in R^{\mathfrak{I}^{\Phi}}$.
        then $\mathfrak{I}^{\Phi} \not\models Rx$ and 
        $\mathfrak{I}^{\Phi} \not\models Ry \Rightarrow \mathfrak{I}^{\Phi} \not\models Rx \lor Ry$
        
    \end{solution}
\end{problem}
\end{document}